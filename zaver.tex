\chap Závěr
    Kontrola plnění hygienických limitů lokální zátěže při repetitivní zátěži je klíčové pro ochranu zdraví pracovníků a předcházení nemocí z~povolání a případným trvalým následkům. Aktuální metodika v~rámci ČR využívá procenta maximální svalové síly vztažené na průměrný směnnový a minutový počet pohybů. Procenta maximální svalové síly jsou zjišťována pomocí \glref {sEMG}, průměrné směnnové a minutové pohyby jsou počítány manuálně na místě, popřípadě za využití videozáznamu.

    Tato práce nejdříve popisuje návrh program, který má za účel nahradit ekosystém od společnosti GETA Centrum s.r.o., který má nedostatky dané z~velké části jeho stářím. Následně je tento program vylepšen metodikou pro vyhledání pohybů přímo v~signálu \glref {EMG}, což může sloužit jako pomůcka pro pracovníky laboratoří fyziologie, při počítání pohybů. Tato metodika kombinuje dvě metody, které se navzájem doplňují, což zařizuje detekci cyklů s~úspěšností 95 \%. Tato chyba od správného počtu zařizuje přesnost v~rámci jednoho procenta Fmax dle tabulky v~příloze Nařízení vlády č. 361/2007 Sb.
