\chap [chap:experiment] Návrh a provedení experimentu

    \sec [sec:experiment-preparation] Příprava experimentu
        Jelikož porovnáváme výsledky z~měření dvou přístrojů, GETA Holter a Shimmer, je třeba změřit co nejsrovnatelněji oba při stejných činnostech. Jelikož sval se experimentem unaví nelze provést měření nejdříve s~jedním přístrojem a následně s~druhým. Další možností bylo provést měření po částech vždy jedním a následně druhým přístrojem. To není ideální z~důvodu nutnosti přepojovat elektrody mezi přístroji, což způsobuje dekalibraci přístroje GETA Holter a zároveň uvolnění elektrod na pokožce, tedy experiment také není validní. Jako nejlepší možnost bylo zvoleno měření zároveň, kdy jsme využili umístění elektrod dle \ref[sec:electrodes-extenzor] a \ref[sec:electrodes-flexor]. Toto umístění bylo mírně poupraveno pro naše potřeby. Na pravé ruce byly elektrody umístěny střídavě, na levé křížem. Toto rozložení bylo zvoleno z~důvodu dosažení co nejvyšší přesnosti, kdy ideální zapojení nelze dosáhnout u~obou, tak byla zvolena tato dvě pro porovnání. U~obou přístrojů byla zemní elektroda přilepena na loket. Po přidělání elektrod a přístrojů GETA Holter a Shimmer byla provedena kalibrace přístroje GETA Holter dle manuálu.

        \midinsert
            \clabel [fig:experiment_ext] {Schéma umístění elektrod při experimentu na dorzální straně}
            \picw=5cm \cinspic experiment_extensor.pdf
            \caption/f Schéma umístění elektrod při experimentu na dorzální straně (oranžově Shimmer, modře GETA Holter). Obrázek převzat od Kateřiny Doubkové\fnotemark1
        \endinsert

        \midinsert
            \clabel [fig:experiment_flex] {Schéma umístění elektrod při experimentu na ventrální straně}
            \picw=5cm \cinspic experiment_flexor.pdf
            \caption/f Schéma umístění elektrod při experimentu na ventrální straně (oranžově Shimmer, modře GETA Holter). Obrázek převzat od Kateřiny Doubkové\fnotemark1
        \endinsert

        \fnotetext{Převzato se souhlasem autorky z~neveřejných zdrojů}

    \sec [sec:experiment] Experiment
        Aktivity v~experimentu byly zvoleny tak, aby byly opakovatelné a jasně měřitelné, ale zároveň co nejvíce se připomínaly činnosti v~praxi při měření na pracovištích. Po každé aktivitě vždy proběhlo 10 sekund odpočinku. Na začátku a na konci experimentu byla změřená referenční $F_{max}$ pomocí dynamometru. Harmonogram experimentu je v~tabulce \ref[tab:experiment]. Měřená osoba při experimentu seděla u~stolu na doraz tak, aby byla schopna položit lokty téměř do pravého úhlu. Aktivity typu přenášení bylo přenášení předmětů o~různých vahách mezi dvěma body vzdálenými 30cm s~položením na začátku a konci pohybu předmětu na stůl do rytmu metronomu nastaveného na 55 bpm. Měřilo se na obou předloktích zároveň.

        \midinsert
            \clabel [tab:experiment] {Harmonogram aktivit v~experimentu}
            \ctable {clc} {
                délka (s) & \hfil činnost                                  & váha (kg) \crl \tskip 4pt
                -         & Měření $F_{max}$ na dynamometru                & -         \cr
                5         & Držení dynamometru na konstantní síle          & 10        \cr
                5         & Držení dynamometru na konstantní síle          & 20        \cr
                5         & Držení dynamometru na konstantní síle          & 30        \cr
                10        & Přenášení 4 puků slepených k~sobě              & 0,65      \cr
                10        & Přenášení hřídele na činku                     & 2,2       \cr
                10        & Přenášení hřídele na činku a kotouč 1,5 kg     & 3,7       \cr
                10        & Přenášení hřídele na činku a kotouč 2,5 kg     & 4,7       \cr
                10        & Přenášení hřídele na činku a 2 kotouče 1,5 kg  & 5,2       \cr
                -         & Rotace zápěstí se závažím (pouze druhý průběh) & 2,2       \cr
                -         & Měření $F_{max}$ na dynamometru                & -         \cr
            }
            \caption/t Harmonogram aktivit v~experimentu
        \endinsert
