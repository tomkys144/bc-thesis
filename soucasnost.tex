\chap Aktuální metodika měření
    \rfc{TODO: Doplnit současný stav}
    V současnosti probíhá měření při každé změně pracovních procesů certifikovanou laboratoří. Měří se každá činnost pro celou průměrnou směnu. Pokud probíhá řízená rotace pracovníků, kdy interval je maximálně jednodenní, pak se pro výpočet užívá časově vážený průměr všech pozic. Měření se provádí minimálně u dvou osob stejného pohlaví, kdy se preferují praváci, kteří byli již dostatečně zapracováni. \cite[mzcr2022vestnik]

    \sec Standardní postup pro provádění měření
        Nejdříve se odmastí povrch pokožky abrazivní pastou, následně se elektrody umístí na předloktí pro měření svalových skupin flexorů a extenzorů, jak jsme popsali v sekcích \ref[sec:electrodes-extenzor] a \ref[sec:electrodes-flexor]. Přístroj i elektrody lze případně připevnit pomocí náplasti či prubanu.

        Po připevnění elktrod se určí velikost Fmax pomocí dynamometru. Měření probíhá v definovaných polohách a to stoj, neutrální poloha v rameni, flexe v lokti 90° a neutrální poloha zápěstí, poté stoj, neutrální poloha v rameni, flexe v lokti 90° a stisk podhmatem, a závěrem stoj a neutrální poloha v ramenu, v lokti i zápěstí. Při určení je palec v opozici, stisk probíhá všemi prsty se zapojením palce. Měření probíhá s minutovým intervalem v každé poloze dvakrát s délkou stisku 2 sekundy. V okamžiku stisku dynamometru nastavíme zesílení přístroje tak, aby hodnota Fmax byla v 1/3 až 2/3 rozsahu, kdy zesílení při hodnotě Fmax musí být stejné jako v průběhu měření.

        Po nastavení zesílení a určení hodnoty Fmax probíhá samotné měření. V jeho průběhu dochází k popisu pracovních činností, zejména časové charakteristiky, odpočinkové časy a určení podílu statické a dynamické složky práce, a záznam počtu pohybů rukou a předloktí. To lze provádět buď na místě, či pomocí videozáznamu. videozáznam lze následně synchronizovat se záznamem EMG, což dává možnost vytipovat rizikové úkony. Počítání pohybů na místě se provádí opakovaně a v náhodných intervalech, kdy na konci se provede aritmetický průměr. Následně se vynásobením průměru hodnot danou výkonnostní normou vypočítá celosměnový počet pohybů. Při počítání se upřednostňuje zjištění počtu pohybů vztažených na 1 úkon, operaci, cyklus nebo kus.V případě že to není možné, lze vycházet z počtu pohybů vztažených na 1 časovou jednotku. Délka měření vždy vychází z požadavku, aby byly vyhodnoceny všechny činnosti prováděné v průměrné pracovní směně. Doporučená délka měření činnosti, kde perioda cyklu nepřekračuje 2 minuty, je minimálně 20 minut. Pokud je perioda delší než 2 minuty, či se střídá větší množství činností, pak se doporučuje měření alespoň 40 minut. \cite[mzcr2022vestnik]

        \midinsert
            \clabel [fig:geta-mereni] {Ukázka křivky v programu EMG Analyzer za užití přístroje EMG Holter}
            \picw=12cm \cinspic GETA-mereni.png
            \caption/f Ukázka křivky v programu EMG Analyzer za užití přístroje EMG Holter
        \endinsert

    \sec Vyhodnocení měření
