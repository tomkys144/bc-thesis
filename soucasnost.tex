\chap Aktuální metodika měření
    \rfc{DONE: Doplnit současný stav}
    V~současnosti probíhá měření při každé změně pracovních procesů certifikovanou laboratoří. Za účelem sjednocení postupů při měření všemi laboratořemi, vydalo Ministerstvo zdravotnictví České republiky ve svém Věstníku 6/2022 Metodický návod k~zajištění jednotného postupu při autorizované měření, posuzování a interpretaci výsledků měření lokální svalové zátěže metodou integrované elektromyografie.

    Podle daného Metodického návodu se měří každá činnost pro celou průměrnou směnu. Pokud probíhá řízená rotace pracovníků, kdy interval je maximálně jednodenní, pak se pro výpočet užívá časově vážený průměr všech pozic. Měření se provádí minimálně u~dvou osob stejného pohlaví, kdy se preferují praváci, kteří byli již dostatečně zapracováni. \cite[mzcr2022vestnik]

    \sec Standardní postup pro provádění měření
        Nejdříve se odmastí povrch pokožky abrazivní pastou, následně se elektrody umístí na předloktí pro měření svalových skupin flexorů a extenzorů, jak jsme popsali v~sekcích \ref[sec:electrodes-extenzor] a \ref[sec:electrodes-flexor]. Přístroj i elektrody lze případně připevnit pomocí náplasti či prubanu.

        Po připevnění elktrod se určí velikost Fmax pomocí dynamometru. Měření probíhá v~definovaných polohách a to stoj, neutrální poloha v~rameni, flexe v~lokti 90° a neutrální poloha zápěstí, poté stoj, neutrální poloha v~rameni, flexe v~lokti 90° a stisk podhmatem, a závěrem stoj a neutrální poloha v~ramenu, v~lokti i zápěstí. Při určení je palec v~opozici, stisk probíhá všemi prsty se zapojením palce. Měření probíhá s~minutovým intervalem v~každé poloze dvakrát s~délkou stisku 2 sekundy. V~okamžiku stisku dynamometru nastavíme zesílení přístroje tak, aby hodnota Fmax byla v~1/3 až 2/3 rozsahu, kdy zesílení při hodnotě Fmax musí být stejné jako v~průběhu měření.

        Po nastavení zesílení a určení hodnoty Fmax probíhá samotné měření. V~jeho průběhu dochází k~popisu pracovních činností, zejména časové charakteristiky, odpočinkové časy a určení podílu statické a dynamické složky práce, a záznam počtu pohybů rukou a předloktí. To lze provádět buď na místě, či pomocí videozáznamu. videozáznam lze následně synchronizovat se záznamem EMG, což dává možnost vytipovat rizikové úkony. Počítání pohybů na místě se provádí opakovaně a v~náhodných intervalech, kdy na konci se provede aritmetický průměr. Následně se vynásobením průměru hodnot danou výkonnostní normou vypočítá celosměnový počet pohybů. Při počítání se upřednostňuje zjištění počtu pohybů vztažených na 1 úkon, operaci, cyklus nebo kus.V případě že to není možné, lze vycházet z~počtu pohybů vztažených na 1 časovou jednotku. Délka měření vždy vychází z~požadavku, aby byly vyhodnoceny všechny činnosti prováděné v~průměrné pracovní směně. Doporučená délka měření činnosti, kde perioda cyklu nepřekračuje 2 minuty, je minimálně 20 minut. Pokud je perioda delší než 2 minuty, či se střídá větší množství činností, pak se doporučuje měření alespoň 40 minut. \cite[mzcr2022vestnik]

        \midinsert
            \clabel [fig:geta-mereni] {Ukázka křivky v~programu EMG Analyzer za užití přístroje EMG Holter}
            \picw=12cm \cinspic GETA-mereni.png
            \caption/f Ukázka křivky v~programu EMG Analyzer za užití přístroje EMG Holter
        \endinsert

    \sec Vyhodnocení měření
        Naměřené hodnoty procent Fmax se časově převáží dle zaznamenaných časových charakteristik na průměrnou směnu. Do časového vážení se nezapočítává zákonná přestávka na jídlo a odpočinek. Bezpečnostní a technologické přestávky se do průměrné směny započítávají také a to hodnotou 5 \% Fmax, vykonával-li pracovník o~přestávce nenáročné drobné operace, a 3 \% nevykonával-li žádnou činnost. Činnosti jako úklid a příprava pracoviště se měřit nemusí, avšak ani v~takovém případě je nelze vynechat z~časového vážení a požívá se hodnota 5 až 8 \% Fmax. Výběr hodnoty provádí odhadem odborný pracovník autorizované fyziologické laboratoře v~závislosti na náročnosti dané činnosti.

        \midinsert
            \clabel [fig:geta-frek] {Ukázka frekvenční analýzy v~programu EMG Analyzer za užití přístroje EMG Holter}
            \picw=8cm \cinspic GETA-frek.png
            \caption/f Ukázka frekvenční analýzy v~programu EMG Analyzer za užití přístroje EMG Holter
        \endinsert

        Převážené průměrné výsledky jsou následně porovnány s~hygienickými limity. Svalové síly nad 70 \% Fmax jsou považovány za nadlimitní zátěž u~prací s~převahou dynamické složky, u~práce s~převahou statické složky je nadlimitní zátěž nad 45 \% Fmax.

        V~případě, že na dané pozici pracují jak muži, tak ženy, nelze výsledek měření přepočítávat. Avšak pokud měřené uskutečněné u~žen nepřekračuje hygienické limity, pak lze výsledek interpretovat tak, že nepřekračuje hygienické limity ani u~mužů. Naopak pokud hotnoty naměřené u~mužů překračují nějaké hygienické limity, pak lze kategorizovat práci u~žen shodně jako u~mužů. V~případech, kdy měření u~žen překročilo hygienické limity, či měření u~mužů limity nepřekročilo, je třeba provést měření samostatně pro obě pohlaví.

        \midinsert
            \clabel [fig:geta-res] {Ukázka výsledků měření v~programu EMG Analyzer za užití přístroje EMG Holter}
            \picw=8cm \cinspic GETA-res2.png
            \caption/f Ukázka výsledků měření v~programu EMG Analyzer za užití přístroje EMG Holter \cite[geta2015analyzer]
        \endinsert

        Při hodnocení lokální svalové zátěže ve směnách delších než 480 minut se provádí navýšení limitů vynakládaných sil v~rozmezí 55-70 \% Fmax a průměrné směnové počty pohybů ruky a předloktí a to vždy o~2.5 \% za každých započatých 30 minut práce nad 480 minut. \cite[mzcr2022vestnik]
