\chap [chap:movement] Vyhledávání pohybů v~signálu EMG
    Jelikož aktuálně se pohyby detekují pouze \rfc{REVIEW: Vysvetlil bych, ze na to kouka ergonom ocitata je za casovy usek} manuálně na místě, popřípadě z~videozáznamu, rozhodl jsem se posunout aktuální metodiku právě v~této oblasti a vytvořit pomůcku pro pracovníky laboratoří fyziologie pro detekci pohybů. Vycházel jsem ze signálů změřených v~průběhu vývoje skriptu v~kapitole \ref[chap:signal], jelikož tato sekce ho doplňuje. Z~těchto měření využívám pouze část s~pohybem puků popsaných v~sekci \ref[sec:experiment]. Délka každého měření byla přibližně 3 minuty. Jelikož se zaměřujeme na prevenci nemocí z~povolání způsobených \glref{DNJZ} končetin, vycházel jsem z~předpokladu, že skript nebude úplně autonomní, avšak bude sloužit pouze jako pomůcka pracovníkovi, který například sám určí množství pohybů v~cyklu, či zvolí přesnější výsledek.

    \rfc{TODO: Na zacatku kapitoly melo but. CO CHCI DELAT. NA JAKYCH DATECH. JAKE BUDU BUDOU VYSLEDKY}
    %  tedy, ze aby bylo mozno to vyhodnotit, tak je treba si to naklikat rucne. Pak vysvetlit postup - ok, to tu je. Jak bych tam pridal i to EKG. Kdyz ne na zacetek. Tak na konec, mozna staci jen jeden snimek finalni segmentovane cervene zeleneho EKG a komentare, ze metoda je obecne pouzitelna na periodicky signaly. To EMG taky vypadat pro kazde pohyby jinak. Na a nakonec jsou vysledky, ty TP, FP. FN.

    Před samotnou detekcí pohybů posunu odečtením prvního vzorku od signálu a tedy v~jsem předešel v~dalších úpravách a filtracích zákmitům (viz obrázek \ref[fig:pohyb-zero]). Následně navrhuji několik metod, pro detekci cyklů v~signálu. Tyto metody jsou popsány v~následujích sekcích.

    \midinsert
        \clabel [fig:pohyb-zero] {Vstupní signál posunutý do nuly}
        \picheight=7cm \cinspic pohyb-zero.png
        \caption/f Vstupní signál posunutý do nuly
    \endinsert

    \sec [sec:corr] Vyhledání opakujících se vzorců
        V~této metodě hledám opakující se cykly přímo z~měřeného signálu. Před samotným vyhledáváním odfiltruji stejnosměrnou složku pomocí inverzního Čebyševova filtru desátého řádu typu horní propust, s~mezním kmitočtem 0,35 Hz a minimálním útlumem v~zádržném pásmu 40 dB. Filtrovaný signál je vykreslen v~obrázku \ref[fig:pohyb-corr-filter]

        \midinsert
            \clabel [fig:pohyb-corr-filter] {Signál filtrovaný horní propustí 0,35 Hz}
            \picheight=7cm \cinspic pohyb-corr-filter.png
            \caption/f Signál filtrovaný horní propustí s~mezním kmitočtem 0,35 Hz
        \endinsert

        Na tomto filtrovaném signálu používám křížovou autokorelaci s~maximálním posunem o~30 sekund dle rovnice \ref[eq:autocorr], kde $x[n]$ je filtrovaný signál a $f_s$ vzorkovací frekvence signálu. \cite[sedlacek1993zpracovani]

        $$
            R_{xx}\left(r\right) = \sum_{n=-30f_s}^{30f_s} x\left[r+n\right]s^*\left[n\right]
            \eqmark[eq:autocorr]
        $$

        Jelikož je signál repetitivní, je možné z~autokorelační funkce odečíst periodu opakující se sekvence, která se ukazuje jako vzdálenost lokálních maxim. Lokální maxima hledám s~minimální vzdáleností 1/4 sekundy, abych předešel nalezení i periody případných šumů. Následně zvolím pouze ty vrcholy, jejichž prominence je větší či rovna průměrné prominenci, což zaručí, že vyberu celou periodu nikoli její část, tedy v~praxi vyberu celý cyklus pohybu tam a zpět, nikoli tam či zpět. To je nežádoucí, jelikož každý z~těchto pohybů může mít jinou sekvenci aktivací svalů v~zápěstí a tedy bych nalezl pouze tento pohyb a ne vždy ty zbylé. Nalezené vrcholy a autokorelační funkce jsou znázorněny v~obrázku \ref[fig:pohyb-corr-xcorr].

        \midinsert
            \clabel [fig:pohyb-corr-xcorr] {Autokorelační funkce signálu s~nalezenými lokálními maximy}
            \picheight=7cm \cinspic pohyb-corr-xcorr.png
            \caption/f Autokorelační funkce signálu s~nalezenými lokálními maximy
        \endinsert

        Díky známé periodě cyklu, je možné vybrat vhodný úsek, který reprezentuje daný cyklus. Toho docílím tak, že naleznu všechna lokální minima, která jsou od sebe vzdálená alespoň 3/4 periody. Následně z~nich vyberu to, které je nejblíže jejich mediánu. Vybraný bod označím jako počátek úseku a konec určím jako bod o~jednu periodu dále. Ukázka takto vybraného úseku je obrázek \ref[fig:pohyb-corr-sample]. Pro výběr lze použít chytřejší algoritmy, či nechat pracovníka laboratoře vybrat manuálně začátek, popřípadě i konec a přeskočit tak potřebu hledání periody, ukázalo se, že tato metoda, je dostatečně přesná.

        \midinsert
            \clabel [fig:pohyb-corr-sample] {Vybraný vzorek reprezentující jednu periodu}
            \picheight=7cm \cinspic pohyb-corr-sample.png
            \caption/f Vybraný vzorek reprezentující jednu periodu
        \endinsert

        Nyní mám reprezentativní vzorek jednoho cyklu a mohu hledat ostatní cykly. Za tím účelem počítám korelaci vybraného vzorku s~pohybujícím se oknem stejné délky jako je vzorek dle rovnice \ref[eq:corr], kde $x$ je úsek vstupního signálu délky $n$, $y$ je referenční vzorek také délky $n$, $\overline{x}$ a $\overline{y}$ jsou průměry daných signálů. \cite[buxton2008correlation] Velikost kroku je jeden bod. Spočítané korelační koeficienty ukazují podobnost daných úseků s~vybraným vzorkem.

        $$
r_{xy} = {{\sum_{k=1}^n\left(x[k] - \overline{x}\right) \left(y[i] - \overline{y}\right)} \over {\sqrt{\sum_{i=1}^n\left(x[i] - \overline{x}\right)^2 \sum_{i=1}^n \left(y[i]-\overline{y}\right)^2}}}
\eqmark[eq:corr]
        $$

        Nyní najdeme lokální maxima korelační koeficientů s~podmínkou, že jsou od sebe vzdálená alespoň 3/4 délky vzorku a z~jejich indexu dopočítáme počáteční indexy nalezených segmentů obsahující detekované cykly.

        \midinsert
            \clabel [fig:pohyb-corr-score] {Korelační koeficient pohybujícícho se okna a vzorku}
            \picheight=7cm \cinspic pohyb-corr-score.png
            \caption/f Korelační koeficient pohybujícícho se okna a vybraného vzorku s~detekovanými lokálními maximy
        \endinsert

        Na závěr vyberu kanál s~lepší detekcí. Zde jsem využíval možnosti výběru pracovníkem laboratoře. V~případě že je nastaven parametr pro interaktivní běh skriptu na {\tt true}, pak pracovník vybere lepší kanál a uložím jeho výběr, v~opačném případě buď nahraji uložené dřívější rozhodnutí obsluhy, či vyberu kanál, kde je více detekovaných cyklů. Následně doplním hluchá místa informacemi z~druhého kanálu. Výsledek je ukázán v~obrázku \ref[fig:pohyb-corr-res]

        \midinsert
            \clabel [fig:pohyb-corr-res] {Výsledek metody vyhledání opakujících se vzorců}
            \picheight=7cm \cinspic pohyb-corr-res.png
            \caption/f Výsledek metody vyhledání opakujících se vzorců
        \endinsert

        Tato metoda má nevýhodu, že ji může rozhodit nízké \glref{SNR}, tedy nízký poměr signálu k~šumu. Nicméně v~takovýchto případech je velmi spolehlivá. Nedostatky této metody se ukazují v~tabulce \ref[tab:pohyb-corr-res] u~subjektů 10 a 11. U~subjektů 5 a 9 byly detekované cykly posunuty o~1/2 periody oproti manuálně označeným hodnotám, nicméně i tato data mohou být v~praxi užitečná pro pracovníka, kde potřebuje počet cyklů, nikoli přesné hranice.

        \midinsert
            \clabel [tab:pohyb-corr-res] {Výsledky metody vyhledání opakujících se vzorců}
            \ctable {l c c c c} {
                Subjekt & TP \hfil & FP\hfil & FN \hfil & Senzitivita (\%) \hfil
                \crl \tskip 4pt
                1 & 328 & 4& 12 & 96,47 \cr
                2 & 680 & 8 & 4 & 99.42 \cr
                3 & 684 & 12 & 0 & 100 \cr
                4 & 316 & 8 & 12 & 96,34 \cr
                5 & 8 & 316 & 324 & 2,41 \cr
                6 & 612 & 8 & 56 & 91,62 \cr
                7 & 288 & 12 & 56 & 83,72 \cr
                8 & 320 & 12 & 20 & 94,12 \cr
                9 & 0 & 332 & 332 & 0 \cr
                10 & 4 & 4 & 536 & 0,74 \cr
                11 & 12 & 0 & 532 & 2,21 \cr
            }
            \caption/t Výsledky metody vyhledání opakujících se vzorců; Volba kanálu proběhla manuálně; TP - Počet spěšně detekovaných cyklů; FP - Počet nepravdivě detekovaných cyklů; FN - Počet nedetekovaných pohybů
        \endinsert

    \sec [sec:emg] Vyhledávání opakujících se vzorců v~obálce signálu
        Jak jsem zmínil, minulá metoda je náchylná na velikost \glref{SNR}. Tuto nevýhodu lze částečně eliminovat vyhledáváním opakujících se vzorců místo v~samotném signálu v~jeho obálce singálu. Pro vyhledání obálky nejdříve přefiltruji signál Čebyševovým filtrem 10. řádu typu pásmová propust s~propustným pásmem 20 - 120 Hz a minimálním útlumem v~zádržném pásmu 40 dB. Tento signál je ukázán v~obrázku \ref[fig:pohyb-emg-envelope]

        Dalším krokem je vyhledání zmíněné obálky. K~tomu využívám \glref{RMS} filtr s~velikostí okna 1/2 sekundy. Dolní obálka je počítána stejně, avšak pro signál vynásobený -1.

        \midinsert
            \clabel [fig:pohyb-emg-envelope] {Signál filtrovaný pásmovou propustí 20 - 120 Hz a jeho obálka}
            \picheight=7cm \cinspic pohyb-emg-envelope.png
            \caption/f Signál filtrovaný pásmovou propustí 20 - 120 Hz a jeho obálka
        \endinsert

        Pro další výpočet používám pouze horní obálku, tedy ji opět posunu do nuly a následně postupuji podobně jako v~sekci \ref[sec:corr], tedy odstraním stejnosměrnou složku horní propustí s~mezní frekvencí 0,35 Hz, naleznu pomocí křížové korelace periodu repetic a vyberu reprezentativní vzorek a najdu všechny jemu podobné úseky. Následně pracovník laboratoře vybere vhodnější kanál a v~něm chybějící informace doplním z~druhého kanálu. Výsledně detekované úseky obálky jsou ukázány v~obrázku \ref[fig:pohyb-emg-envelope-res].

        \midinsert
            \clabel [fig:pohyb-emg-envelope-res] {Výsledek detekce opakujících se úseků obálky}
            \picheight=7cm \cinspic pohyb-emg-envelope-res.png
            \caption/f Výsledek detekce opakujících se úseků obálky
        \endinsert

        Tato metoda je značně spolehlivější u~signálu s~vyšším šumem, nicméně u~činností, kde je nízká vynaložená síla ztrácí dostatek dat. To lze pozorovat v~tabulce \ref[tab:pohyb-emg-res] u~subjektů 1 a 2. Naopak subjekty 10 a 11 mají na rozdíl od tabulky \ref[tab:pohyb-corr-res] velmi vysokou úspěšnost.

        \midinsert
            \clabel [tab:pohyb-emg-res] {Výsledky metody vyhledávání opakujících se vzorců v~obálce signálu}
            \ctable {l c c c c} {
                Subjekt & TP \hfil & FP\hfil & FN \hfil & Senzitivita (\%) \hfil
                \crl \tskip 4pt
                1 & 124 & 232 & 216 & 36,47 \cr
                2 & 320 & 32 & 368 & 46,51 \cr
                3 & 636 & 68 & 72 & 89,83 \cr
                4 & 312 & 24 & 16 & 95,12 \cr
                5 & 324 & 12 & 12 & 96,43 \cr
                6 & 568 & 100 & 112 & 83,53 \cr
                7 & 252 & 116 & 116 & 68,48 \cr
                8 & 336 & 0 & 4 & 98,82 \cr
                9 & 328 & 4 & 4 & 98,8 \cr
                10 & 540 & 0 & 0 & 100 \cr
                11 & 544 & 0 & 0 & 100 \cr
            }
            \caption/t Výsledky metody vyhledávání opakujících se vzorců v~obálce signálu; Volba kanálu proběhla manuálně; TP - Počet spěšně detekovaných cyklů; FP - Počet nepravdivě detekovaných cyklů; FN - Počet nedetekovaných pohybů
        \endinsert

    \sec [sec:merge_methods] Spojení metod
        Jednitlivé dříve popsané metody mají své výhody a nevýhody a jsou úspěšnější pro jiný typ signálu. Díky tomu je třeba opět vybrat lepší metodu a následně doplnit případná chybějící data pomocí zbylých metod. K~tomu používám podobný způsob rozhodování jako při výberu lepších kanálů, kdy pracovník laboratoře buď vybere manuálně lepší metodu a jeho výběr se uloží pro příští spuštění skriptu na tom samé datasetu, popřípadě se použije metoda s~vyšším počtem cyklů. Detekované cykly po spojení jsou znázorněny v~obrázku \ref[fig:pohyb-res].

        \midinsert
            \clabel [fig:pohyb-res] {Detekované cykly pomocí navržené metodiky}
            \picheight=7cm \cinspic pohyb-result.png
            \caption/f Detekované cykly pomocí navržené metodiky
        \endinsert

        Kombinace těchto metod s~manuální volbou jak kanálů, tak metod, je velmi efektivní a dosahuje specificity 95 \%, jak je ukázáno v~tabulce \ref[tab:pohyb-res]. Ukazuje se také, že subjekty 1, 4 a 6 se zlepšili oproti tabulce \ref[tab:pohyb-corr-res] díky využití výsledků z~metody vyhledávání opakujících se vzorců na obálce.

        \midinsert
            \clabel [tab:pohyb-res] {Výsledky metodiky}
            \ctable {lcccc} {
                Subjekt & TP \hfil & FP\hfil & FN \hfil & Senzitivita (\%) \hfil
                \crl \tskip 4pt
                1 & 348 & 16 & 0 & 100 \cr
                2 & 680 & 8 & 4 & 99,42 \cr
                3 & 688 & 24 & 0 & 100 \cr
                4 & 324 & 8 & 4 & 98,78 \cr
                5 & 332 & 20 & 8 & 97,65 \cr
                6 & 696 & 20 & 12 & 98,31 \cr
                7 & 348 & 40 & 20 & 94,57 \cr
                8 & 336 & 4 & 4 & 98,82 \cr
                9 & 328 & 8 & 4 & 98,8 \cr
                10 & 540 & 0 & 0 & 100 \cr
                11 & 544 & 0 & 0 & 100 \cr
            }
            \caption/t Výsledky metodiky s~kombinací metod \ref[sec:corr] a \ref[sec:emg]; Volba kanálů i metod proběhla manuálně; TP - Počet spěšně detekovaných cyklů; FP - Počet nepravdivě detekovaných cyklů; FN - Počet nedetekovaných pohybů
        \endinsert

    \sec [sec:other_signal] Využití při měření jiných signálů
    Navrženou metodiku jsem zkusil aplikovat i na jiné signály \rfc{TODO: Tady bych to vubec rozsiril a diskuze o~vyhodach a nevyhodach metody. Kde to fungovat bude a kde to fungovat nebude}, než pouze EMG a to na EKG, konkrétně na dataset ECG-ID. \cite[lugovaya2005biometric] Ukazuje se, že metodika by při ladění jednotlivých parametrů, jako je vzdálensot vrcholů, či jednotlivé prahy měla fungovat. \rfc{TODO: Divna formulace, by mohla pri ladeni parametru. Ukazal bych treba vysledky jako to dopadlo plne automaticky, a pak vysledky jako do dopadlo vyladene} Výsledek je ukázán v~obrázku \ref[fig:ecg].


        \midinsert
            \clabel [fig:ecg] {Výsledek aplikace metodiky na signál ECG}
            \picheight=7cm \cinspic ecg.png
            \caption/f Výsledek aplikace metodiky na signál ECG
        \endinsert
