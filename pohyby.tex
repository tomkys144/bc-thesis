\chap Vyhledávání pohybů v~signálu EMG
    Jelikož aktuálně se pohyby detekují pouze manuálně na místě, popřípadě z~videozáznamu, rozhodl jsem se posunout aktuální metodiku právě v~této oblasti a vytvořit pomůcku pro pracovníky laboratoří fyziologie pro detekci pohybů. Vycházel jsem ze signálů změřených v~průběhu vývoje skriptu v~kapitole \ref[chap:signal], jelikož tato sekce ho doplňuje. Z~těchto měření využíváme pouze část s~pohybem puků popsaných v~sekci \ref[sec:experiment].Délka každého měření byla přibližně 3 minuty. Jelikož se zaměřujeme na prevenci nemocí z~povolání způsobených \glref[DNJZ] končetin, hledám opakující se cykly.

    Samotný signál nejdříve posuneme tak, aby začínal v~0
    (viz obrázek \ref[fig:pohyb-zero]) a následně využijeme několik metod, pro detekci signálu. Tyto metody jsou popsány v~následujích sekcích.

    \midinsert
        \clabel [fig:pohyb-zero] {Vstupní signál posunutý do nuly}
        \picheight=7cm \cinspic pohyb-zero.png
        \caption/f Vstupní signál posunutý do nuly
    \endinsert

    \sec Hledání podobných vzorců
        V~této metodě hledáme opakující se cykly přímo z~měřeného signálu. Před samotným vyhledáváním odfiltrujeme stejnosměrnou složku pomocí inverzního Čebyševova filtru desátého řádu typu horní propust, s~mezním kmitočtem 0.35 Hz a minimálním útlumem v~zádržném pásmu 40 dB. Přenosová funkce toho filtru je zakreslena v~obrázku \ref[fig:pohyb-corr-hp]. Filtrovaný signál je vykreslen v~obrázku \ref[fig:pohyb-corr-filter]

        \midinsert
            \clabel [fig:pohyb-corr-hp] {Přenosová funkce horní propusti 0.35 Hz}
            \picheight=7cm \cinspic pohyb-corr-hp.png
            \caption/f Přenosová funkce použité horní propusti s~mezním kmitočtem 0.35 Hz
        \endinsert

        \midinsert
            \clabel [fig:pohyb-corr-filter] {Signál filtrovaný horní propustí 0.35 Hz}
            \picheight=7cm \cinspic pohyb-corr-filter.png
            \caption/f Signál filtrovaný horní propustí s~mezním kmitočtem 0.35 Hz
        \endinsert

        Na tomto filtrovaném signálu používáme křížovou autokorelaci s~maximálním posunem o~30 sekund. Jelikož je signál repetitivní, je možné z~autokorelační funkce odečíst periodu opakující se sekvence, která se ukazuje jako vzdálenost lokálních maxim. Tedy nalezneme pomocí lokální maxima s~minimální vzdáleností 0.25 sekundy, abychom předešli nalezení pouze jemných zvlnění. Následně zvolíme pouze ty vrcholy, jejichž prominence je větší než průměrná, což zaručí, že vybereme celou periodu nikoli její část, tedy v~praxi vybereme celý cyklus pohybu tam a zpět, nikoli tam či zpět. To je nežádoucí, jelikož každý z~těchto pohybů využívá jinak svaly v~zápěstí a tedy bychom nenalezli jeho opak. Nalezené vrcholy a autokorelační funkce jsou znázorněny v~obrázku \ref[fig:pohyb-corr-xcorr].

        \midinsert
            \clabel [fig:pohyb-corr-xcorr] {Autokorelační funkce signálu s~nalezenými lokálními maximy}
            \picheight=7cm \cinspic pohyb-corr-xcorr.png
            \caption/f Autokorelační funkce signálu s~nalezenými lokálními maximy
        \endinsert
