\label [chap:signal]

\chap Zpracování signálu

    \sec Zpracování signálu z~přístroje Shimmer
        Shimmer ukládá data do souborů csv. Je tedy třeba je načíst a vybrat korektní sloupce (18 a 20) vytvořit z~nich vektory.

        \midinsert
            \clabel [fig:signal_raw] {Surová data z~přístroje Shimmer}
            \picheight=7cm \cinspic signal_raw.png
            \caption/f Surová data z~přístroje Shimmer
        \endinsert

        Následně filtrujeme šumy a artefakty ze signálu. Volíme \glref{BP} na frekvencích 20 - 120 Hz. Pásmová zádrž není třeba použít díky napájení přístroje z~baterií. Využíváme butterworthův filter 6. řádu.

        \midinsert
            \clabel [fig:signal_bp] {Signál filtrovaný pásmovou propustí}
            \picheight=7cm \cinspic signal_bp.png
            \caption/f Signál filtrovaný pásmovou propustí
        \endinsert

        Následně spočítáme obálku signálu. Ta se rovná absolutní hodnotě signálu.

        \midinsert
            \clabel [fig:signal_envelope] {Obálka filtrovaného signálu}
            \picheight=7cm \cinspic signal_envelope.png
            \caption/f Obálka filtrovaného signálu
        \endinsert

        Tato obálka lze použít, pro porovnání se GETA je třeba převzorkovat signál ze vzorkovací frekvence 256 Hz na 1 Hz. To provedeme spočítáním \glref{RMS} dle vzorce \ref[eq:rms] pro každých 256 vzorků (tedy jednu sekundu signálu).

        $$
            x_{RMS} = \sqrt{{1 \over 256} \sum_{i=1}^{256} x_i^2}
            \eqmark[eq:rms]
        $$

        \midinsert
            \clabel [fig:signal_reduced] {Převzorkovaná obálka filtrovaného signálu}
            \picheight=7cm \cinspic signal_reduced.png
            \caption/f Převzorkovaná obálka filtrovaného signálu
        \endinsert

        Nyní můžeme porovnat náš zpracovaný signál z~přístroje Shimmer s~výsledkem z~"blackboxu" systému GETA. Vidíme, že po normalizaci průměrnou hodnotou, jsou si signály podobné. Odchylky lze očekávat vzhledem k~neideálním podmínkám měření.

        \midinsert
            \clabel [fig:signal_compare] {Srovnání signálů mezi přístroji}
            \picheight=7cm \cinspic signal_compare.png
            \caption/f Srovnání signálů mezi přístroji
        \endinsert

    \sec Vytvoření histogramů
        Histogramy hledáme pro násobky 5 \% maximální síly. Nejdříve spočítáme prahy jednotlivých binů a následně přiřadíme hodnoty počtu vyskytujících se vzorků mezi prahy do binů. Hodnoty větší než je maximální síla zahrnujeme do binu větší než 95 \%. Následně pak normalizujeme velikosti histogramu tak, aby sčítal do 100.

        \midinsert
            \clabel [fig:histogram] {Histogram měření EMG}
            \picheight=7cm \cinspic histogram.png
            \caption/f Histogram měření EMG
        \endinsert
