\chap Zpracování signálů

\sec Zpracování signálu z~přístroje Shimmer

Shimmer ukládá data do souborů csv. Je tedy třeba je načíst a vybrat korektní sloupce (18 a 20) vytvořit z~nich vektory.

\begtt
data = readtable(shimmer_file);
channel1 = table2array(data(:, 18));
channel2 = table2array(data(:, 20));
\endtt

\midinsert
\clabel [fig:signal_raw] {Surová data z~přístroje Shimmer}
\picheight=8cm \cinspic signal_raw.png
\caption/f Surová data z~přístroje Shimmer
\endinsert

Následně filtrujeme šumy a artefakty ze signálu. Volíme \glref{BP} na frekvencích 20 - 120 Hz. Pásmová zádrž není třeba použít díky napájení přístroje z~baterií. Využíváme butterworthův filter 6. řádu.

\begtt
high_pass_freq = 20;
low_pass_freq = 120;

[b, a] = butter(6, [high_pass_freq / fs_shimmer * 2 ...
    low_pass_freq / fs_shimmer * 2]);

channel1_bp = filter(b, a, channel1);
channel2_bp = filter(b, a, channel2);
\endtt

\midinsert
\clabel [fig:signal_bp] {Signál filtrovaný pásmovou propustí}
\picheight=8cm \cinspic signal_bp.png
\caption/f Signál filtrovaný pásmovou propustí
\endinsert


Následně spočítáme obálku signálu. Ta se rovná absolutní hodnotě signálu.

\begtt
envelope1=channel1_abs;
envelope2=channel2_abs;
\endtt

\midinsert
\clabel [fig:signal_envelope] {Obálka filtrovaného signálu}
\picheight=8cm \cinspic signal_envelope.png
\caption/f Obálka filtrovaného signálu
\endinsert

Tato obálka lze použít, pro porovnání se GETA je třeba převzorkovat signál ze vzorkovací frekvence 256 Hz na 1 Hz. To provedeme spočítáním \glref{RMS} dle vzorce \ref[eq:rms] pro každých 256 vzorků (tedy jednu sekundu signálu).

$$
x_{RMS} = \sqrt{{1 \over 256} \sum_{i=1}^{256} x_i^2}
\eqmark[eq:rms]
$$

\begtt
signal = envelope;
num_samples = length(signal);
reduced_signal = arrayfun(@(i) ...
    rms(signal(i:min(i + fs_shimmer - 1, num_samples))), ...
    1:fs_shimmer:num_samples)';
\endtt

\midinsert
\clabel [fig:signal_reduced] {Převzorkovaná obálka filtrovaného signálu}
\picheight=8cm \cinspic signal_reduced.png
\caption/f Převzorkovaná obálka filtrovaného signálu
\endinsert

Nyní můžeme porovnat náš zpracovaný signál z~přístroje Shimmer s~výsledkem z~"blackboxu" systému GETA. Vidíme, že po normalizaci průměrnou hodnotou, jsou si signály podobné. Odchylky lze očekávat vzhledem k~neideálním podmínkám měření.

\midinsert
\clabel [fig:signal_compare] {Srovnání signálů mezi přístroji}
\picheight=8cm \cinspic signal_compare.png
\caption/f Srovnání signálů mezi přístroji
\endinsert

\sec Vytvoření histogramů

Histogramy hledáme pro násobky 5\% maximální síly. Nejdříve spočítáme prahy jednotlivých binů a následně přiřadíme hodnoty počtu vyskytujících se vzorků mezi prahy do binů. Hodnoty větší než je maximální síla zahrnujeme do binu větší než 95%. Následně pak normalizujeme velikosti histogramu tak, aby sčítal do 100.

\begtt
wanted = linspace(0, 95, 95/5+1);
wanted = [wanted, 1000];
thresholds = wanted / 100 * f_max;

bins = zeros(size(wanted));

for i = 1:length(signal)

    for j = 1:length(thresholds) - 1

        if (signal(i) >= thresholds(j) && signal(i) < thresholds(j + 1))
            bins(j) = bins(j) + 1;

            continue
        end

    end

end

max_value = max(bins) / 100;

bins = bins ./ sum(bins) * 100;

wanted(wanted == 1000) = 100;
\endtt

\midinsert
\clabel [fig:histogram] {Histogram měření EMG}
\picheight=8cm \cinspic histogram.png
\caption/f Histogram měření EMG
\endinsert
