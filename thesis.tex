% arara: vlna: { all: true }
% arara: optex
% arara: optex
% arara: optex

% !TEX root = ./thesis.tex

\input ctustyle3
\input dirtree

\def \montht {
    \ifcase \month \or Leden \or Únor \or Březen \or Duben \or Květen \or Červen \or Červenec \or Srpen \or Září \or Říjen \or Listopad \or Prosinec \fi
}
\def \fmark#1{$^\mathbox{#1}$}
% \setff{+frac}\currvar

\draft
% \savetoner

\worktype [B/CZ]
\faculty {F3}
\department {Katedra teorie obvodů}

\title {Analýza EMG při hodnocení lokální fyzické zátěže}
\titleEN{EMG analysis in the assessment of local physical load}

\author {Tomáš Kysela}

\studyinfo {Lékařská elektronika a bioinformatika}

\date {\montht \_the\_year}

\supervisor{Ing. Jaromír Doležal, Ph.D.}

\specification {
    \vbox to0pt{\vskip-25mm\centerline{\inspic zadani.pdf }\vss}
}

\thanks {
    Tímto bych chtěl poděkovat především vedoucímu mé práce panu Ing.~Jaromíru Doležalovi,~Ph.D., a to za ochotu se pravidelně a často scházet ke konzultacím, podporu při řešení všech potíží spojených s~psaním bakalářské práce a za velmi milý přístup.

    Dále bych chtěl poděkovat Ing.~Jindřichovi Adolfovi, za ochotu a přístup při provádění experimentů a konzultace při porovnávání přístrojů Shimmer a GETA.

    Dále velké poděkování patří paní doc.~Ing.~Lence Lhotské,~CSc. za zprostředkování této práce a Ing.~Iloně Kačerové,~Ph.D. za ochotu a rady při měření a poskytnutí přístroje GETA.

    Na závěr bych chtěl poděkovat rodině a přátelům, kteří mne v~průběhu psaní této práce nemálo podporovali.
}

\declaration {
    Prohlašuji, že jsem předloženou práci vypracoval samostatně a že jsem uvedl veškeré použité informační zdroje v~souladu s~Metodickým pokynem o~dodržování etických principů při přípravě vysokoškolských závěrečných prací.

    V~Praze dne \_the\_day.~\_the\_month.~\_the\_year
    \signature
}

\abstractCZ {
    Detekce rizikových pracovních procesů je klíčová pro prevenci onemocnění z~povolání spojených s~nadměrnou dlouhodobou jednostrannou zátěží končetin. Tato práce zkoumá současnou metodiku měření pracovníků za účelem detekce rizikových procesů pomocí elektromyografie, z~které se počítá vynaložené síly. Ta se spolu s~počtem pohybů užívá k~porovnání s~hygienickými limity. Následně navrhuje novou poloautomatizovanou metodu detekce pohybových cyklů, kterou ověřuje na naměřených experimentálních datech.
}
\abstractEN {
    Detection of high-risk work processes is crucial in preventing occupational diseases associated with prolonged one-sided strain on limbs. This thesis examines the current methodology of measuring workers to detect such processes using electromyography, from which exerted force is calculated. This force, in relation to a number of movements, is used in determining hygienic limits. Then, it proposes a new semi-automated method for detecting movement cycles, validated using experimental data.
}

\keywordsCZ {
    EMG, detekce pohybů, MATLAB, fyziologie práce
}
\keywordsEN {
    EMG, movement detection, MATLAB, work physiology
}

\input glosdata

\picdir={assets/}

\makefront

\input uvod

\input svaly

\input emg

\input legislativa

\input soucasnost

\input experiment

\input signal

\input pohyby

\input zaver

\input prilohy

\bye
