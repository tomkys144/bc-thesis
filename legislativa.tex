\chap Legislativa v~oblasti pracovní ergonomie

    \sec Česká republika
        Hygienické limity u~fyzické zátěže stanoví v~České republice nařízení vlády č. 361/2007 Sb., konkrétně hlava IV.

        Její první díl se zabývá celkovou fyzickou zátěží, neboli zátěž při dynamické fyzické práci vykonávané velkými svalovými skupinami, při které je zatěžováno více než 50 \% svalové hmoty. Limity jsou stanoveny za užití energetického výdeje, viz tabulky \ref[tab:energy-man] a \ref[tab:energy-woman], a srdeční frekvence, která nesmí průměrně přesáhnout 102 tepů za minutu a jednorázově 110 tepů za minutu. Obě tyto hodnoty platí pouze není li okamžitá hodnota vyšší než 28 tepů za minutu oproti klidové. \cite[2007narizeni]

        \midinsert
            \clabel [tab:energy-man] {Přípustné a průměrné hygienické limity energetického výdeje při práci s~celkovou fyzickou zátěží u~mužů}
            \ctable {l c c c c} {
                Energetický výdej & 15 - 16 let \hfil & 16 - 17 let \hfil & 17 - 18 let \hfil & 18 a více let \hfil
                \crl \tskip 4pt
                Směnový průměrný [MJ] & 5,9 & 6,9 & 7,9 & 6,8 \cr
                Směnový přípustný [MJ] & 6,2 & 7,3 & 8,5 & 8 \cr
                Roční průměrný [MJ] & 1390 & 1620 & 1860 & 1600 \cr
                Minutový přípustný [W] & 440 & 500 & 540 & 575 \cr
            }
            \caption/t Přípustné a průměrné hygienické limity energetického výdeje při práci s~celkovou fyzickou zátěží u~mužů \cite[2007narizeni]
        \endinsert

        \midinsert
            \clabel [tab:energy-woman] {Přípustné a průměrné hygienické limity energetického výdeje při práci s~celkovou fyzickou zátěží u~žen}
            \ctable {l c c c c} {
                Energetický výdej & 15 - 16 let \hfil & 16 - 17 let \hfil & 17 - 18 let \hfil & 18 a více let \hfil
                \crl \tskip 4pt
                Směnový průměrný [MJ] & 3,7 & 3,8 & 4,8 & 4,5 \cr
                Směnový přípustný [MJ] & 4,4 & 4,6 & 5,0 & 5,4 \cr
                Roční průměrný [MJ] & 870 & 890 & 1130 & 1060 \cr
                Minutový přípustný [W] & 350 & 370 & 375 & 395 \cr
            }
            \caption/t Přípustné a průměrné hygienické limity energetického výdeje při práci s~celkovou fyzickou zátěží u~žen \cite[2007narizeni]
        \endinsert

        Druhý díl se pak zabývá zatížením lokálním. Limity při lokálním zatížení, tedy zatížení malých svalových skupin při práci končetinami, jsou posuzovány na základě počtu pohybů vztažených k~průměrné časově vážené hodnotě vynakládaných svalových sil vyjádřené v~procentech maximální svalové síly (Fmax). Tento vztah je vyjádřen v~tabulce \ref[tab:movement]. \cite[2007narizeni]

        \midinsert
            \clabel [tab:movement] {Průměrné hygienické limity pro směnové a minutové počty pohybů ruky a předloktí za průměrnou osmihodinovou směnu}
            \ctable {c c c } {
                \% Fmax & Průměrný počet pohybů & Průměrný minutový počet pohybů \crl \tskip 4pt
                7 & 27600 & 58 \cr
                8 & 24300 & 51 \cr
                9 & 21800 & 44 \cr
                10 & 19800 & 41 \cr
                11 & 18100 & 37 \cr
                12 & 16700 & 34 \cr
                \vdots & \vdots & \vdots \cr
                48 & 3200 & 7 \cr
                49 & 3000 & 7 \cr
                50 & 2700 & 7 \cr
                51 & 2400 & 7 \cr
                52 & 2100 & 7 \cr
                53 & 1800 & 7 \cr
            }
            \caption/t Průměrné hygienické limity pro směnové a minutové počty pohybů ruky a předloktí za průměrnou osmihodinovou směnu \cite[2007narizeni]
        \endinsert

    \sec Slovenská republika
        Ve Slovenské republice je povinnost dodržovat hygienické limity udává zákon č. 355/2007 Z. z. o~ochrane, podpore a rozvoji verejného zdravia, přesněji §30 a §38. Podle nich je zaměstnavatel povinen zajistit posouzení fyzické zátěže při práci, dodržovat nejvyšší přípustné hodnoty vynakládaných svalových sil a frekvence pohybů a dodržovat nejvyšší přípustné hodnoty celkové a lokální zátěže zaměstnanců. Hodnocení se provádí zejména u~prací malých svalových skupin horních končetin s~vysokými počty pohybů, kde se porovnává podobně jako v~ČR počet pohybů vztažených k~průměrné časově vážené hodnotě vynakládaných svalových sil vyjádřené v~procentech maximální svalové síly. \cite[halasova2022posuzovani, 2007zakon]
