\chap Úvod
    Lokální nadměrná zátěž razantně zvyšuje únavu konkrétních svalů. To může následně způsobovat riziko některých úrazů. Mezi tyto úrazy se řadí mimo jiné syndrom karpálního tunelu, tenisový loket, tendinitida, či bursitida. \cite[hecht2017repetitive] Za účelem předcházení výskytů těchto zranění mezi pracovníky v~průmyslu bylo vydáno Nařízení vlády č. 361/2007 Sb. Pro kontrolu pracovních procesů a jejich vyhovění legislativě \rfc{DONE: Přeformulovat} se využívá přístroje EMG Holter se softwarovým zpracováním v~programu EMG Analyzer, oboje od společnosti GETA Centrum s.r.o., která má v~oblasti měření fyziologie práce na našem území prakticky monopol.

    V~první části je vysvětleno fungování přístrojů pro měření \glref{EMG}. Nejdříve je vysvětleno fungování neuromuskulárního systému a následně jeho obecné zpracování.

    V~druhé části práce je vysvětleno stávající řešení od společnosti GETA a následně shrnuto dané Nařízení vlády.

    Ve třetí části se pak zabýváme návrhem softwaru pro zpracování signálu z~přístroje Shimmer. Nejdříve navrhujeme metodu zpracování, na jejímž konci máme stejné výsledky jako stávající řešení, aby šlo nový přístroj využít ve stávajících procesech měřících laboratoří. Následně se pokusíme nalézt nové informace za využití detailnějšího signálu z přístroje Shimmer, konkrétně detekci repetitivních vzorů a tedy opakujících se cyklů pohybu.
