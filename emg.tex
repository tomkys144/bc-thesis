% !TEX root = ./thesis.tex

\chap [chap:emg] Signál EMG

    \sec [sec:emg-hw] Součásti přístroje pro měření EMG
        Přístroje pro měření \glref{EMG} měří změny potenciálu v~jednotlivých svalech. Specificita měření se může velmi lišit, od měření jednotlivých vláken, přes svaly, až po celé svalové skupiny. To záleží primárně na použité elektrodě a na filtrovaných frekvencích. Samotný přístroj je v~podstatě voltmetr, který měří napětí velká řádově milivolty a následně je zesiluje na vhodné velikosti a potlačuje šum. Cesta signálu je poté z~elektrod přes diferenciální zesilovač a filtry do AD převodníku.

        \secc [sec:emg-hw-electrodes] Elektrody
            Elektrody jsou způsob jak interagují elektrodiagnostické metody s~lidským tělem. Dají se dělit na dva hlavní typy: povrchové a invazivní.

            Povrchové elektrody začínaly jako měděné plošky či kroužky, dnes jsou tyto znovupoužitelné elektrody nahrazovány jednorázovými za účelem snížení rizika infekce. Existuje několik typů povrchových elektrod: elektrody s~přímým kontaktem, plovoucí elektrody, hydrogelové elektrody a páskové elektrody.

            Elektrody s~přímým kontaktem bývaly dříve často měděné a připevněny za pomoci leukoplasti. Jejich velkou výhodou je citlivost u~slabých signálů \glref{sEMG}, tedy například měření klidových svalových činností, naopak se nehodí pro měření dynamických pohybů z~důvodu omezení pohybu a jejich odlepování.

            Plovoucí elektrody jsou téměř přesným opakem elektrod s~přímým kontaktem. Samotná elektroda je zavěšena v~malém kalíšku přibližně 1 mm nad pokožkou. Jejich nevýhodou je náročnost přípravy měření a nižší citlivost. Avšak výhodou je nízké omezení pohybu a tak jsou vhodné na dynamické měření.

            Dnešním standardem jsou elektrody hydrogelové. Tyto elektrody jsou vyráběné z~chloridu stříbrného a jsou přidělány slabou vrstvou vodivého hydrogelu. Díky němu mají nižší impedanci, což způsobuje vyšší šum. Jsou podobné elektrodám s~přímým kontaktem, a tak jsou i vhodná na podobné měření, ale drží často lépe, je tedy možné je použít i na málo dynamické pohyby. Zároveň hydrogel je analergický, tedy vhodný pro pacienty s~citlivou pokožkou.

            Invazivní elektrody jsou hlavně dvou typů: bipolární a monopolární elektrody. Bipolární elektroda jsou prakticky dvě elektrody v~jednom. Ve středu je drátková elektroda typicky z~platiny s~povrchem standardně velikosti mezi $0{,}01 \hbox{ a } 0{,}09 \hbox{ mm}^2$, typicky ${0{,}07} \hbox{ mm}^2$. Následně je obalená izolační vrstvou, a poté nerezovým povrchem který slouží jako druhá elektroda. Monopolární je naopak pouze nerezová jehla s~potahem z~teflonu s~odhaleným 1-5 mm hrotu, který slouží jako elektroda o~ploše cca $0{,}03 - 0{,}34 \hbox{ mm}^2$.\cite[criswell2011cram, tankisi2020standards]

        \secc [sec:emg-hw-amp] Zesilovač
            V~přístrojích pro měření EMG se využívá diferenciální zesilovač. To znamená, že nezesiluje napětí na vstupech, ale rozdíl těchto napětí. Toto zesílení se nazývá gain. V~ideálním zesilovači je zesílen pouze rozdíl. To však v~reálném zesilovači nelze, společnou složku pouze potlačuje. Kvalitu tohoto potlačení určuje parametr \glref{CMRR} daný vztahem \ref[eq:cmrr].

            $$
                \hbox{CMRR}_{\hbox{dB}} = 20\times\log_{10}({\hbox{differential-mode gain} \over \hbox {common-mode gain}})
                \eqmark[eq:cmrr]
            $$

            \glref{CMRR} je často kolem 120-150 dB při 50 Hz a s~vyššími frekvencemi pak klesá.

            Vstupy zesilovače jsou 3, 2 aktivní a jeden pasivní. Dnes jsou značeny E1 (černý vstup), E2 (červený vstup) a E0 (zelený vstup). Dříve se vyskytovalo značení G1, G2 a ground, či active, reference a ground. \cite[criswell2011cram, tankisi2020standards]

            \midinsert
                \clabel [fig:diff-amp] {Schéma fungování diferenciálního zesilovače}
                \picw 10cm \cinspic diff-amp.png
                \caption/f Schéma fungování diferenciálního zesilovače; A) Rozdílové zesílení; B) Souhlasné zesílení \cite[tankisi2020standards]
            \endinsert

        \secc [sec:emg-hw-filters] Analogové filtry
            V~zesíleném signálu se stále i po potlačení společné složky vyskytují šumy a artefakty. Artefakty jsou dvou typů, technologické a biologické. Mezi technologické se řadí cable motion artefakt (frekvence 1-10 Hz), šum z~nedokonalého spojení elektrody s~pokožkou danou převážně roztahováním pokožky či z~biomedicínských zařízení (jako je např. kardiostimulátor). Mezi biologické se řadí převážně šum z~okolních svalů (tzv. crosstalk). Tyto šumy je následně třeba odfiltrovat, či alespoň minimalizovat. \cite[criswell2011cram, tankisi2020standards]

            Základní charakteristikou všech filtrů je přenosová funkce definovaná vztahem \ref[eq:hs], kde $s=j\omega$ a $\omega$ je úhlová frekvence. Jelikož cívky a kondenzátory mají impedanci závislou na frekvenci, je i přenosová funkce závislá na frekvenci vstupu a udává útlum při dané frekvenci.

            $$
                H(s) = {V_{out} \over V_{in}} = {\sum_{m=0}^M a_m s^m \over \sum_{n=0}^N b_n s^n}
                \eqmark[eq:hs]
            $$

            Dalším parametrem kterým se při návrhu řídíme je mezní kmitočet. Ten je definován jako kmitočet, při kterém útlum dosahuje 3 dB. Od něj se následně odvíjí zádržné a propustné pásmo, kdy propustné pásmo je takový frekvenční rozsah, kde útlum je menší než 3 dB, popřípadě méně, a zádržné, kde je větší útlum. Typicky však vyžadujeme útlum větší než 3 dB, tedy mezi zádržné a propustné pásmo vkládáme ještě přechodné. Pokud je propustné pásmo na nižších frekvencích než zádržné, pak se jedná o~filtr typu dolní propust. Jeho protikladem je horní propust. Tyto filtry lze kaskádově skládat za sebe, kdy v~případě skládání filtrů stejného typu vytváříme strmější, popřípadě silnější filtry, zatímco složením filtrů rozdílných typů umíme vytvořit pásmovou propust, popřípadě zádrž. \cite[analogdevicesfilters]

            Analogové filtry působí ve spojitém čase a lze je často snadno vytvořit pomocí diskrétních součástek, jako jsou kondenzátory, rezistory, cívky či operační zesilovače. Zde se setkáváme s~dělením na pasivní a aktivní, kdy pasivní filtr se skládá pouze z~pasivních prvků. Typickým pasivním filtrem je tzv RLC filtr složený z~rezistorů, cívek a kondenzátorů. Aktivní filtr pak využívá nejen rezistorů a kondenzátorů, ale i součástky vyžadující externí napájení, jako je operační zesilovač, typicky se zpětnou vazbou. \cite[young2008designing, analogdevicesfilters]

            Tyto filtry pak mají několik základních topologií. První z~nich je tzv Butterwortův filtr. Tento filtr není zvlněn v~propustném ani zádržném pásmu. Jeho nevýhodou pak je široká přechodná zóna. Přesným opakem je Cauerův filtr, anglicky elliptic filter, který obětuje plochou přenosovou funkci za úzkou přechodnou zónu. Mezi nimi je tzv. Čebyševův filtr (někdy označován jako typ 1), který je zvlněn pouze v~zádržném pásmu a následně má užší přechodnou zónu. Jeho inverze, označována také jako typ 2, je pak zvlněna v~zádržném pásmu nikoli v~propustném. Porovnání přenosových funkcí je v~obrázku \ref[fig:filtr-compare]. \cite[young2008designing, analogdevicesfilters]

            \midinsert
                \clabel [fig:filtr-compare] {Porovnání topologií filtrů}
                \picw 10cm \cinspic filtr-compare.png
                \caption/f Porovnání topologií filtrů
            \endinsert

        \secc [sec:adc] AD převodník
            Pro digitální zpracování je třeba analogový signál diskretizovat. Toho dosahujeme pomocí AD převodníku, který přiřazuje analogovému signálu v~daných časech diskrétní hodnotu. Pro zachování kvality signálu je třeba mít dostatečné rozlišení a vysokou vzorkovací frekvenci. Na druhou stranu zbytečně vysoké rozlišení a frekvence způsobí zbytečně velké využití paměti.

            Nyquistův teorém říká, že vzorkovací frekvence by měla být alespoň dvakrát tak velká, jak nejvyšší sledovaná frekvence. V~praxi je nejčastěji dvakrát až pětkrát vyšší, než nejvyšší sledovaná frekvence. Typicky používané frekvence v~\glref{EMG} jsou v~tabulce \ref[tab:filtfreq]. V~této tabulce jsou uváděny hodnoty, které zachovávají všechny sledované informace, avšak různé využití \glref{EMG} využívají jiných informací, což způsobuje, že často jsou využívány nižší vzorkovací frekvence než uváděny v~tabulce.

            Rozlišení je třeba stanovit takové, aby se neztratily sledované změny v~signálu. Obecně uznávané minimum je 8 bitů, tedy 256 hodnot, jelikož jeden bit vyjadřuje znaménko. Dnes se často používají 24 bitové AD převodníky, které umí vyjádřit 16~777~216 hodnot. \cite[tankisi2020standards]

            \midinsert
                \clabel [tab:filtfreq] {Typické hodnoty amplitudy, frekvencí BP a vzorkovací frekvence}
                \caption/t Typické hodnoty amplitudy, frekvencí \glref{BP} a vzorkovací frekvence \cite[tankisi2020standards]

                \ctable {lccc} {
                    \hfil Metoda měření & Amplituda         & Frekvence \glref{BP}              & Vzorkovací frekvence \crl \tskip 4pt
                    Jehlové \glref{EMG} & $0-30 \hbox{ mV}$ & $2 \hbox{ Hz} - 10 \hbox{ kHz}$   & $50 \hbox{ kHz}$     \cr
                    \glref{sfEMG}       & $0-50 \hbox{ mV}$ & $500 \hbox{ Hz} - 10 \hbox{ kHz}$ & $50 \hbox{ kHz}$     \cr
                    \glref{sEMG}        & $0-10 \hbox{ mV}$ & $1 \hbox{ Hz} - 1 \hbox{ kHz}$    & $5 \hbox{ kHz}$      \cr
                }
            \endinsert

    \sec [sec:electrodes] Umístění elektrod
        Existuje mnoho možných poloh elktrod. Obecně je třeba vybrat vhodné místo pro E0, E1 a E2 a umístit je tak, aby mezi nimi byla měřená oblast, tedy typicky podél svalu, či svalové skupiny, popřípadě na opačné končetiny. Jelikož se v~rámci práce zabývám pouze měřením u~předloktí, uvedu zde možné umístění elektrod pro měření svalů předloktí.

        \secc [sec:electrodes-flex_ext] Měření flexoru a extenzoru předloktí
            Toto umístění slouží k~sledování celkového pnutí v~předloktí. Využívá se toho k~analýze zranění horní končetiny, např. z~repetitivních činností.

            První elektrodu umisťujeme nad extenzor zápěstí, tedy na dorzální stranu předloktí přibližně 5 cm od lokte. Vhodné místo nalezneme pohmatem, konkrétně položíme prsty na přibližnou lokaci a při extenzi zápěstí se jedná o~střed zvýrazněné svalové skupiny.

            Druhá elektroda se umisťuje nad flexor zápěstí, tedy ventrální stranu předloktí přibližně 5 cm od lokte. Místo nalezneme podobně jako u~první elektrody, pouze zápěstí tentokrát přivedeme do flexe.

            Samotné měření je náchylné na činnost ostatních svalových skupin horní končetiny a pronaci/supinaci zápěstí. Měřené hodnoty mohou být také ovlivněny polohou horní končetiny, kdy hodnota při složení rukou v~sedě bude jiná, než při rukách volně visících podél těla při stání. \cite[criswell2011cram]

            \midinsert
                \clabel [fig:flex-extenzor] {Schéma umístění elektrod pro měření flexoru a extenzoru předloktí}
                \picw=5cm \cinspic flex_ext.pdf
                \caption/f Schéma umístění elektrod pro měření flexoru a extenzoru předloktí. Obrázek převzat od Kateřiny Doubkové\fnotemark1; Data převzata z~\cite[criswell2011cram]
            \endinsert

            \fnotetext{Převzato se souhlasem autorky z~neveřejných zdrojů}

        \secc [sec:electrodes-extenzor] Měření extenzorů zápěstí
            Při tomto umístění měříme extenzory zápěstí, primárně extensor digitorum, dále pak extensor carpi radialis a extensor carpi ulnaris. Slouží k~posouzení činnosti extenzorů zápěstí za účelem předcházení a léčby zranění způsobených repetitivními činnostmi.

            Elektrody se umisťují nad extenzor zápěstí, tedy na dorzální stranu předloktí, přibližně 5 cm od lokte. Přesné umístění zjistíme pohmatem při extenzi zápěstí, kdy elektrody se umísťují do středu zvýrazněné svalové skupiny, 3-4 cm od sebe ve směru svalových vláken.

            Je třeba brát na vědomí, že hodnoty mohou být ovlivněny polohou paže, zápěstí a prstů, a úrovní pronace/supinace zápěstí. \cite[criswell2011cram]

            \midinsert
                \clabel [fig:extenzor] {Schéma umístění elektrod pro měření extenzorů předloktí}
                \picw=3cm \cinspic extensor.pdf
                \caption/f Schéma umístění elektrod pro měření extenzorů předloktí. Obrázek převzat od Kateřiny Doubkové\fnotemark1; Data převzata z~\cite[criswell2011cram]
            \endinsert

            \fnotetext{Převzato se souhlasem autorky z~neveřejných zdrojů}

        \secc [sec:electrodes-flexor] Měření flexoru zápěstí
            Tímto umístěním měříme flexory zápěstí pro sledování činnosti flexorů zápěstí při prevenci a léčbě zranění zápěstí.

            Elektrody umístíme na flexory zápěstí, tedy ventrální stranu předloktí, přibližně 5 cm od lokte. Přesnou lokaci nalezneme pohmatem při flexi zápěstí, kdy elektrody dáváme do středu zvýrazněné svalové skupiny přibližně 3-4 cm od sebe ve směru svalových vláken.

            Hodnoty mohou být ovlivněny polohou a podepřením ruky, prstů a paže, a úrovní pronace/supinace zápěstí. \cite[criswell2011cram]

            \midinsert
                \clabel [fig:flexor] {Schéma umístění elektrod pro měření flexorů předloktí}
                \picw=3cm \cinspic flexor.pdf
                \caption/f Schéma umístění elektrod pro měření flexorů předloktí. Obrázek převzat od Kateřiny Doubkové\fnotemark1; Data převzata z~\cite[criswell2011cram]
            \endinsert

            \fnotetext{Převzato se souhlasem autorky z~neveřejných zdrojů}

    \sec [sec:emg-model] Modelování signálu EMG
        Jedním ze základních přístupů k~modelování signálu \glref{EMG} je zvažovat zvažovat jednotlivé akční potenciály motorických neuronů samostatně, bez ovlivnění od okolních a šumů, které se následně spojíme pomocí konvoluční funkce. V~modelu vyvozeném Sherifem a Gregorem v~roce 1986 se předpokládá, že průměrný počet nervových impulzů je závislý na čase a síle vyvinuté svalem. Pro jendotlivé tkáně a přístroje se vytváří přenosové funkce, nakonec se přidává i šum. Schéma takového modelu je ukázáno v~obrázku \ref[fig:sherif-gregor-model]. Generovaný signál EMG takovýmto modelem lze popsat rovnicí \ref[eq:emgst], zatímco změřený signál rovnicí \ref[eq:emgwt], kde $h(t)$ je přenosová funkce tkání, $K_1$ a $K_2$ konstanty, $r(t)$ přenosová funkce přístrojů a $n_1(t)$ s~$n_2(t)$ jsou šumové funkce. \cite[nieminen1989methods]

        $$
            S\left(t\right) = n_1 \left(t\right) + \int_0^t K_1 \cdot h\left(t-\tau\right) \cdot V\left(\tau\right) d\tau
            \eqmark[eq:emgst]
        $$

        $$
            W\left(t\right) = n_2 \left(t\right) + \int_0^t K_2 \cdot r\left(t-\tau\right) \cdot S\left(\tau\right) d\tau
            \eqmark[eq:emgwt]
        $$

        \midinsert
            \clabel [fig:sherif-gregor-model] {Schéma modelu pro generování EMG signálu}
            \picw 8cm \cinspic sherif-gregor-model.png
            \caption/f Schéma modelu pro generování EMG signálu \cite[nieminen1989methods]
        \endinsert

        Lindström následně odvodil rovnici pro modelování výkonového spektra \glref{EMG} v~homogenním médiu. DOkázal, že spektrum závisí na tvaru a rychlosti vedení akčního potenciálu, součtu všech přispívajících signálů ve motorickém neuronu a umístění elektrod. Všechny tyto vlastnosti vyjadřuje funkce $\bi G \mit\left({\omega \over \nu}\right)$, kde $\omega$ je úhlová frekvence a $\nu$ rychlost akčního potenciálu. Výkonové spektrum pak vyjadřuje rovnice \ref[eq:w-omega]. Tento model navíc spojuje spektrální posuv, který se vyskytuje v~při únavě, přímo s~nižší rychlostí vedení akčních potenciálů. \cite[nieminen1989methods]

        $$
            \bi W \mit = \nu^{-2} \cdot \bi G \mit \left({\omega \over \nu}\right)
            \eqmark[eq:w-omega]
        $$

        Dalším možným přístupem je možné považovat signál \glref{EMG} za stochastický výstup systému řízeného Gaussovským šumem. Pro lineární fiting se pak používají různé modely, například AR (autoregresivní), ARMA (autoregresivní klouzavý průměr) či ARIMA (autoregresivní integrovaný klouzavý průměr). Hlavní výhodou této metody je popis všech důležitých vlastností \glref{EMG} signálu pomocí lineárního modelu. Tyto metody se uplatní primárně pro klasifikaci segmentů signálu \glref{EMG}.

    \sec [sec:emg-sw] Digitální zpracování EMG

        \secc [sec:emg-sw-filters] Digitální filtry
            Oproti analogovým filtrům popsaných v~sekci \ref[sec:emg-hw-filters] digitální filtry pracují se signálem, který byl již diskretizován pomocí \glref{ADC} (viz sekce \ref[sec:adc]). Díky tomu lze dosáhnout i filtrací, které nejsou v~analogovém světě možné. Základní dělení digitálních filtrů je dle jejich impulzní odezvy a to na filtry s~konečnou impulzní odezvou (\glref{FIR}) a nekonečnou impulzní odezvou (\glref{IIR}).

            \glref{FIR} filtry mají výstup závislý pouze na aktuálních a minulých hodnotách vstupního signálu. Jejich matematický popis je dán rovnicí \ref[eq:fir], princip je znázorněn obrázkem \ref[fig:fir]. Díky tomu mají lineární fázový posun, tedy všechny frekvence jsou posunuty stejně a zachovávají se fázové charakteristiky vstupního signálu. Další výhodou je jejich stabilita nezávisle na vstupním signálu. \cite[sarpal2020difference, hwe2023fir]

            $$
                y(n) = \sum_{k=0}^N b_k \cdot x(n-k)
                \eqmark[eq:fir]
            $$

            \midinsert
                \clabel [fig:fir] {Schéma obecného FIR filtru}
                \picw 7cm \cinspic fir.png
                \caption/f Schéma obecného FIR filtru \cite[hwe2023fir]
            \endinsert

            Oproti tomu \glref{IIR} filtry využívají zpětnou vazbu a tedy k~výpočtu využívají i předchozí výstupní hodnoty. Výstupní hodnota je pak dána rovnicí \ref[eq:iir], což lze znázornit obrázkem \ref[fig:iir]. Jejich výhodou je primárně nižší využití paměti, nižší zpoždění či možnost analogové implementace. Naopak je třeba si dát pozor na nelineární fázový posun a hlavně na stabilitu systému. Zpětná vazba může systém rozkmitat a ačkoli vstup byl konečně velký, výstup může nabývat nekonečných hodnot. Stabilitu systému lze vyhodnotit pomocí koeficientů $a_k$, kde pro koeficienty uvnitř jednotkové kružnice, tedy $|a_k| < 1$ je systém stabilní. \cite[sarpal2020difference, hwe2023fir]

            $$
                y(n) = \sum_{k=0}^P b_k \cdot x(n-k) - sum_{k=1}^Q a_k \cdot y(n-k)
                \eqmark[eq:iir]
            $$

            \midinsert
                \clabel [fig:iir] {Schéma obecného IIR filtru}
                \picw 5cm \cinspic iir.png
                \caption/f Schéma obecného IIR filtru \cite[hwe2023fir]
            \endinsert

        \secc [sec:emg-corr] Křížová korelace signálu
            Křížová korelace je matematická metoda pro určení podobnosti dvou signálů. Matematický předpis křížové korelace dvou na sobě nezávislých signálů $x(t)$ a $y(t)$ je dán rovnicí \ref[eq:corr_rt], resp. \ref[eq:corr_rm] pro diskrétní signály, kde $x^*$ značí komplexní sdružení.

            $$
                R_{xy}(\tau) = \int_{-\infty}^\infty x(t) y^*(t-\tau) dt
                \eqmark[eq:corr_rt]
            $$

            $$
                R_{xy}[m] = \sum_{n=-\infty}^\infty x[n] y^*[n-m] \eqmark[eq:corr_rm]
            $$

            V~praxi to znamená, že postupně posouváme signál o~$m$ a spočítáme průnik ploch pod grafem. Výsledkem je funkce, ze které lze odečíst nejen podobnost signálu, ale například zpoždění signálu, které se projeví jako první lokální maximum této funkce, jelikož v~daném posunutí si budou funkce nejpodobnější.

            Pokud jsou oba signály stejné, nazýváme tuto operaci autokorelací. Zde při posunutí 0 dostáváme vždy maximální korelaci, avšak v~případě periodického signálu objeví se v~autokorelační funkci další lokální maxima. Zde jsme schopni zjistit délku periody ze vzdálenosti jednotlivých maxim. \cite[sneha2017understanding]

            \midinsert
                \clabel [fig:corr] {Grafické znázornění korelace}
                \picw 7cm \cinspic correlation.png
                \caption/f Grafické znázornění korelace; Převzato z~\cite[cmglee2016visual]
            \endinsert

        \secc Obálka signálu
            Obálkou signálu je funkce, která tvoří ohraničení signálu. Díky tomu lze získat mnoho informací, mezi něž patří demodulace amplitudou modulovaného signálu, či vyhlazení šumů u~nízkofrekvenčních signálů.

            Existuje několik definic obálky, zejména pak analytická, či \glref{RMS}. Analytická obálka využívá Hilbertovu transformaci. Ta je definována jako konvoluce signálu se signálem $1 \over \pi t$, tedy

            $$
                \hat{g}(t) = \mscrH\left[g(t)\right] = g(t) * {1 \over \pi t} = {1 \over \pi} \int_{-\infty}^{\infty} {g\left(t-\tau\right) \over \tau} d\tau
                \eqmark
            $$

            Z~ní následně dostáváme analytickou obálku $m(t)$ danou rovnicí \ref[eq:analytic-env]. \cite[kschischang2015hilbert, yang2017signal]

            $$
                m(t) = \left[g^2(t) + \hat{g}^2(t)\right]^{1/2}
                \eqmark[eq:analytic-env]
            $$

            Další variantou je obálka získaná pomocí interpolací lokálních extrémů funkce. K~tomu se obecně užívá interpolace spline křivkou druhého řádu. \cite[yang2017signal]

            Poslední variantou je využití \glref{RMS} filtru. Obálka v~čase $t$ je pak hodnota \glref{RMS} okna velikosti $2T+1$ soustředěného kolem času $t$. Taková obálka je dána přepisem \ref[eq:rms-env].

            $$
                m(t) = \left[{1 \over 2T + 1} \int_{t-T}^{t+T} g^2(t)\right]^1/2
                \eqmark[eq:rms-env]
            $$

            \midinsert
                \clabel [fig:env] {Porovnání obálek}
                \picheight 7cm \cinspic envelope.png
                \caption/f Porovnání různých metod získání obálek
            \endinsert
