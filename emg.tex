\chap Princip fungování EMG

\sec Neuromuskulární systém

Neuromuskulární systém je základem pohybového aparátu člověka. Patří do něj kosterní svalstvo a nervy, které je ovládají.

\secc Kosterní svalstvo

Každý sval se skládá ze svalových vláken, která jsou shlukována do svazků. Každý svazek je následně držen pojivovou tkání. Samotné svazky poté nejsou často vedeny pouze jedním směrem, nýbrž jsou různě pootočeny tak, aby součet vektorů jejich sil dohromady tvořil požadovaný vektor.

Každé vlákno se následně skládá z~ještě menších dílků zvaných myofibrily. Myofibrily jsou obaleny sarkopazmatickým retikulem a jsou invaginovány T-tubulami. Každá myofirbrila se následně skládá z~tlustého a tenkého filamentu.

Tlustý filament je tvořen myosinem. Ten se skládá z~šesti polypeptidů, kdy 2 tvoří pár těžkých řetězců a 4 tvoří dva páry lehkých řetězců. Těžký řetězec je převážně stočen do alfa-šroubovice, kdy tvoří ocásek myosinové molekule. Na konci každého z~těžkých řetězců spolu s~párem lehkých řetězců poté tvoří globulární myosinové hlavy.

Tenký filament se skládá převážně z~aktinu. Aktin v~tenkém filamentu je polymerizován do dvou vláken stočených do alfa-šroubovice. Na této šroubovici jsou místa k~vázání myosinu. Tato místa jsou při relaxaci zakrytá tropomyosinem. Na něm jsou v~pravidelných intervalech zavěšené komplexy troponinu. Jeho úkolem je při kontrakci navázat ionty vápníku a odstranit tropomyosin a dovolit navázání myosinových hlav na aktin.

Každá myofibrila je pruhované vlákno kdy jednotlivé sekce se nazývají sarkomery. V~prostřed sarkomery se vyskytuje A-pásmo. Zde se prolínají aktin a myosin. Ve středu A-pásma se nachází M-linie. Ty jsou tvořeny tmavě zbarvenými proteiny vázajícími jednotlivé molekuly myosinu k~sobě. Na hranicích sarkomery pak leží I-pásmo obsahující aktin. Uprostřed každého I-pásma se nachází Z-disk který ohraničuje konce jednotlivých sarkomer. \cite[costanzo2018physiology]

\midinsert
\clabel [fig:sarkomera] {Schéma sarkomery}
\picw=7.5cm \cinspic assets/sarkomera.png
\caption/f Schéma sarkomery \cite[luther2009vertebrate]
\endinsert

\secc Motorické neurony

Ve svalu by nikdy neprobíhala kontrakce nebýt motorických neuronů. Motorické neurony jsou nervové buňky, které slouží k~přenášení impulzů z~kortexu mozku a mozkového kmene ke svalu. Dělí se na dva typy, horní a dolní.
Horní motorické neurony jsou součástí \glref{CNS} a vedou signál z~kortexu mozku, mozkového kmene a mozečku míchou k~jednotlivým dolním motorickým neuronům.

Dolní motorické neurony jsou poté nervové buňky, které mají za úkol přenášet signál od horních nervových neuronů. Existují tři hlavní typy dolních motorických neuronů: somatické motorické neurony, branchiální motorické neurony a viscerální motorické neurony. Somatické motorické neurony se dále dělí na tři podtypy: alfa, beta a gamma. Alfa motorické neurony inervují extrafuzální svalová vlákna a jsou primárními nosiči vzruchu při kontrakci kosterních svalů. Jejich těla leží v~mozkovém kmeni či v~míše. Gamma motorické neurony naopak inervují svalová vřeténka a určují jejich citlivost. \cite[zayia2023neuroanatomy]

\secc Kontrakce a relaxace svalového vlákna

V~klidu jsou na myosinových hlavách připevněné molekuly \glref{ADP} a \glref{$\hbox{P}_i$}. Ty jsou záporně nabité. Stejně tak jsou záporně nabitá vlákna aktinu a tím pádem se myosin s~aktinem slabě odpuzují. \cite[criswell2011cram]

\midinsert
\clabel [fig:kontrakce] {Kontrakce svalového vlákna}
\picw=7.5cm \cinspic assets/Skeletal_Muscle_Contraction.jpg
\rfc{Nechat?}
\caption/f Kontrakce svalového vlákna \cite[whitneymenefee2020human]
\endinsert

Kontrakce začíná přijetím nervového akčního potenciálu, který se z~dolních motorických neuronů šíří do T-tubul. Depolarizace T-tubul způsobí otevření $\hbox{Ca}^{++}$ kanálků v~sarkoplazmatickém retikulu. Ty vypouštějí ionty $\hbox{Ca}^{++}$ které se navazují na troponin na tenkých filamentech, což posune tropomyosinem a odhalují se místa k~vázání myosinu. Nyní začíná tzv. cross-bridge cyklus. Vypuštění \glref{ADP} a \glref{$\hbox{P}_i$} a myosinová hlava se přichytává k~aktinu a následně se posouvá směrem k~M-linii. To způsobuje pohyb aktinu a posun o~cca 10 nm. Následně se na hlavu přichytává molekula \glref{ATP}, která se štěpí na \glref{ADP} a \glref{$\hbox{P}_i$}. Energie z~reakce narovnává myosinovou hlavu do původní polohy a cyklus může začít znovu. \cite [costanzo2018physiology,criswell2011cram, whitneymenefee2020human]

Při relaxaci pak dochází ke snížení koncentrace $\hbox{Ca}^{++}$ za pomoci ATP-poháněných pump, kterého ionty odčerpávají zpět do sarkoplazmatického retikula, což způsobí opětovné navázání tropomyosinu na aktin a myosinová hlava se nemůže přichytit. Následně stejná polarita aktinu s~\glref{ADP} způsobuje odsunutí filamentů do počáteční polohy. \cite [costanzo2018physiology,criswell2011cram, whitneymenefee2020human]

\sec Hardwarové zpracování signálu EMG

\rfc{vymyslet možná lepší název}
Přístroje pro měření \glref{EMG} měří změny potenciálu v~jednotlivých svalech. Mohou být buď hodně obecné, či měřit konkrétní svalová vlákna. To záleží primárně na použité elektrodě a na filtrovaných frekvencích. Samotný přístroj je v~podstatě voltmetr, který měří napětí velká řádově milivolty a následně je zesiluje na vhodné velikosti a potlačuje šum. Cesta signálu je poté z~elektrod přes diferenciální zesilovač a filtry do AD převodníku.

\secc Elektrody

Elektrody jsou způsob jak interagují elektrodiagnostické metody s~lidským tělem. Dají se dělit na dva hlavní typy: povrchové a invazivní.

Povrchové elektrody začínali jako měděné plošky či kroužky, dnes jsou tyto znovupoužitelné elektrody nahrazovány jednorázovými za účelem snížení rizika infekce. Existuje několik typů povrchových elektrod: elektrody s~přímým kontaktem, plovoucí elektrody, hydrogelové elektrody a páskové elektrody.

Elektrody s~přímým kontaktem jsou často měděné a drží připevněny za pomoci leukoplasti. Jejich velkou výhodou je citlivost u~slabých signálů \glref{sEMG}, tedy například měření klidových svalových činností, naopak se nehodí pro měření dynamických pohybů z~důvodu omezení pohybu a jejich odlepování.

Plovoucí elektrody jsou téměř přesným opakem elektrod s~přímým kontaktem. Samotná elektroda je zavěšena v~malém kalíšku přibližně 1 mm nad pokožkou. Jejich nevýhodou je náročnost přípravy měření a nižší citlivost. Výhodou však je nízké omezení pohybu a tak jsou vhodné na dynamické měření.

Dnešním standardem jsou elektrody hydrogelové. Tyto elektrody jsou vyráběné z~chloridu stříbrného a jsou přidělány slabou vrstvou vodivého hydrogelu. Díky němu mají nižší impedanci, což způsobuje vyšší šum. Jsou podobné elektrodám s~přímým kontaktem a tak jsou i vhodná na podobné měření, ale drží často lépe, tedy je možné je použít i na málo dynamické pohyby. Zároveň hydrogel je analergický, tedy vhodný pro pacienty s~citlivou pokožkou.

Invazivní elektrody jsou poté hlavně dvou typů: bipolární a monopolární elektrody. Bipolární elektroda jsou prakticky dvě elektrody v~jednom. Ve středu je drátková elektroda typicky z~platiny s~povrchem standardně velikosti mezi 0.01 a 0.09 $\hbox{mm}^2$, typicy $0.07 \hbox{ mm}^2$. Následně je obalená izolační vrstvou a následně nerezovým povrchem který slouží jako druhá elektroda. Monopolární je naopak pouze nerezová jehla s~potahem z~teflonu s~odhaleným 1-5 mm hrotu, který slouží jako elektroda o~ploše cca $0.03 - 0.34 \hbox{ mm}^2$.\cite[criswell2011cram, tankisi2020standards]

\secc Zesilovač

V~přístrojích pro měření EMG se využívá diferenciální zesilovač. To znamená, že nezesiluje napětí na vstupech, ale rozdíl těchto napětí. Toto zesílení se nazývá gain. V~ideálním zesilovači je zesílen pouze rozdíl. To však v~reálném zesilovači nelze, společnou složku pouze potlačuje. Kvalitu tohoto potlačení určuje parametr \glref{CMRR} daný vztahem \ref[eq:cmrr].

$$
\hbox{CMRR}_{\hbox{dB}} = 20\times\log_{10}({\hbox{differential-mode gain} \over \hbox {common-mode gain}})
\eqmark[eq:cmrr]
$$
\glref{CMRR} je často kolem 120-150 dB při 50 Hz a s~vyššími frekvencemi pak klesá.

Vstupy zesilovače jsou 3, 2 aktivní a jeden pasivní. Dnes jsou značeny E1 (černý vstup), E2 (červený vstup) a E0 (zelený vstup). Dříve se vyskytovalo značení G1, G2 a ground, či "active", "reference" a "ground". \cite[criswell2011cram, tankisi2020standards]

\secc Filtry

V~zesíleném signálu se stále i po potlačení společné složky vyskytují šumy a artefakty. Artefakty jsou dvou typů technologické a biologické. Mezi technologické se řadí "cable motion artefakt" (frekvence 1-10 Hz), šum z~nedokonalého spojení elektrody s~pokožkou danou převážně roztahováním pokožky či z~biomedicínských zařízení (jako je např. kardiostimulátor). Mezi biologické se řadí převážně šum z~okolních svalů (tzv. "crosstalk"). Tyto šumy je následně třeba odfiltrovat, či alespoň minimalizovat. Za tímto účelem se používá několik typů filtrů.

Pásmová zádrž se používá k~odstranění artefaktů vzniklých interferencí 50 Hz z~napájení. Tato zádrž je velmi úzká, tak aby odstranila pouze 50 Hz a jeho násobky, typicky široká jednotky Hz.

Horní propust odstraňuje nižší frekvence. To odstraňuje pomalu se měnící signály, zkreslení průběhu signálu a hlavně snižuje zpoždění k~vrcholům průběhu. Naopak může sama způsobit vznik artefaktů z~filtrování.

Dolní propust naopak filtruje vysoké frekvence za účelem snížení amplitudy, šumu a snižuje náběžný čas.

Obě tyto propusti se poté nejčastěji spojují do pásmové propusti. Nejčastěji používané hodnoty jsou zaznamenány v~tabulce \ref[tab:filtfreq]. \cite[criswell2011cram, tankisi2020standards]

\secc AD převodník

Pro digitální zpracování je třeba analogový signál diskretizovat. Toho dosahujeme pomocí AD převodníku, který přiřazuje analogovému signálu v~daných časech diskrétní hodnotu. Pro zachování kvality signálu je třeba mít dostatečné rozlišení a vysokou vzorkovací frekvenci. Na druhou stranu zbytečně vysoké rozlišení a frekvence způsobí zbytečně velké využití paměti.

Nyquistův teorém říká, že vzorkovací frekvence by měla být alespoň dvakrát tak velká, jak nejvyšší sledovaná frekvence. V~praxi je nejčastěji dvakrát až pětkrát vyšší, než nejvyšší sledovaná frekvence. Typicky používané frekvence v~\glref{EMG} jsou v~tabulce \ref[tab:filtfreq].

Rozlišení je třeba stanovit takové, aby se neztratili sledované změny v~signálu. Obecně uznávané minimum je 8 bitů, tedy 256 hodnot, jelikož jeden bit vyjadřuje znaménko. Dnes se často používají 24 bitové AD převodníky, které umí vyjádřit 16~777~216 hodnot. \cite[tankisi2020standards]

\midinsert
\clabel [tab:filtfreq] {Typické hodnoty amplitudy, frekvencí BP a vzorkovací frekvence}
\ctable {lccc} {
    \hfil Metoda měření & Amplituda & Frekvence \glref{BP} & Vzorkovací frekvence \crl \tskip 4pt
    Jehlové \glref{EMG} & $0-30 \hbox{ mV}$ & $2 \hbox{ Hz} - 10 \hbox{ kHz}$ & $50 \hbox{ kHz}$ \cr
    \glref{sfEMG} & $0-50 \hbox{ mV}$ & $500 \hbox{ Hz} - 10 \hbox{ kHz}$ & $50 \hbox{ kHz}$ \cr
    \glref{sEMG} & $0-10 \hbox{ mV}$ & $1 \hbox{ Hz} - 1 \hbox{ kHz}$ & $5 \hbox{ kHz}$ \cr
}
\caption/t Typické hodnoty amplitudy, frekvencí \glref{BP} a vzorkovací frekvence \cite[tankisi2020standards]
\endinsert

\sec Umístění elektrod

Existuje mnoho možných poloh elktrod. Obecně je třeba vybrat vhodné místo pro E0, a E1 a E2 umístit tak, aby mezi nimi byla měřená oblast, tedy typicky podél svalu, či svalové skupiny, popřípadě na opačné končetiny. Jelikož se v~rámci práce zabývám pouze měřením u~předloktí, uvedu zde možné umístění elektrod pro měření svalů předloktí.

\secc Měření flexoru a extenzoru předloktí

Toto umístění slouží k~sledování nespecifického pnutí v~předloktí. Využívá se toho k~analýze zranění horní končetiny, např. z~repetitivních činností.

První elektrodu umisťujeme nad extenzor zápěstí, tedy na dorzální stranu předloktí přibližně 5 cm od lokte. Vhodné místo nalezneme pohmatem, konkrétně položíme prsty na přibližnou lokaci a při extenzi zápěstí se jedná o~střed zvýrazněné svalové skupiny.

Druhá elektroda se umisťuje nad flexor zápěstí, tedy ventrální stranu předloktí přibližně 5 cm od lokte. Místo nalezneme podobně jako u~první elektrody, pouze zápěstí tentokrát přivedeme do flexe.

Samotné měření je náchylné na činnost ostatních svalových skupin horní končetiny a pronaci/supinaci zápěstí. Měřené hodnoty mohou být také ovlivněny polohou horní končetiny, kdy hodnota při složení rukou v~sedě bude jiná, než při rukách volně visících podél těla při stání. \cite[criswell2011cram]

\midinsert
\clabel [fig:flex-extenzor] {Schéma umístění elektrod pro měření flexoru a extenzoru předloktí}
\picw=5cm \cinspic assets/flex_ext.png
\caption/f Schéma umístění elektrod pro měření flexoru a extenzoru předloktí \cite[criswell2011cram]
\endinsert

\secc Měření extenzorů zápěstí

Při tomto umístění měříme extenzory zápěstí, primárně extensor digitorum, dále pak extensor carpi radialis a extensor carpi ulnaris. SLouží k~posouzení činnosti extenzorů zápěstí za účelem předcházení a léčby zranění způsobených repetitivními činnostmi.

Elektrody se umisťují nad extenzor zápěstí, tedy na dorzální stranu předloktí přibližně 5 cm od lokte. Přesné umístění zjistíme pohmatem při extenzi zápěstí, kdy elektrody se umísťují do středu zvýrazněné svalové skupiny, 3-4 cm od sebe ve směru svalových vláken.

Je třeba brát na vědomí, že hodnoty mohou být ovlivněny polohou paže, zápěstí a prstů a úrovní pronace/supinace zápěstí. \cite[criswell2011cram]

\midinsert
\clabel [fig:extenzor] {Schéma umístění elektrod pro měření extenzorů předloktí}
\picw=3cm \cinspic assets/extensor.png
\caption/f Schéma umístění elektrod pro měření extenzorů předloktí \cite[criswell2011cram]
\endinsert

\secc Měření flexoru zápěstí

Tímto umístěním měříme flexory zápěstí pro sledování činnosti flexorů zápěstí při prevenci a léčbě zranění zápěstí.

Elektrody umístíme na flexory zápěstí tedy ventrální stranu předloktí přibližně 5 cm od lokte. Přesnou lokaci nalezneme pohmatem při flexi zápěstí, kdy elektrody dáváme do středu zvýrazněné svalové skupiny přibližně 3-4 cm od sebe ve směru svalových vláken.

Hodnoty mohou být ovlivněny polohou a podepřením ruky, prstů a paže, a úrovní pronace/supinace zápěstí. \cite[criswell2011cram]

\midinsert
\clabel [fig:flexor] {Schéma umístění elektrod pro měření flexorů předloktí}
\picw=3cm \cinspic assets/flexor.png
\caption/f Schéma umístění elektrod pro měření flexorů předloktí \cite[criswell2011cram]
\endinsert
