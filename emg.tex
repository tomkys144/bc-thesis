\chap Princip fungování EMG

\sec Neuromuskulární systém
Neuromuskulární systém je základem pohybového aparátu člověka. Patří do něj kosterní svalstvo a nervy, které je ovládají.

\secc Kosterní svalstvo
Každý sval se skládá ze svalových vláken, která jsou shlukována do svazků. Každý svazek je následně držen pojivovou tkání. Samotné svazky poté nejsou často vedeny pouze jedním směrem, nýbrž jsou různě pootočeny tak, aby součet vektorů jejich sil dohromady tvořil požadovaný vektor.

Každé vlákno se následně skládá z~ještě menších dílků zvaných myofibrily. Myofibrily jsou obaleny sarkopazmatickým retikulem a jsou invaginovány T-tubulami. Každá myofirbrila se následně skládá z~tlustého a tenkého filamentu.

Tlustý filament je tvořen myosinem. Ten se skládá z~šesti polypeptidů, kdy 2 tvoří pár těžkých řetězců a 4 tvoří dva páry lehkých řetězců. Těžký řetězec je převážně stočen do alfa-šroubovice, kdy tvoří ocásek myosinové molekule. Na konci každého z~těžkých řetězců spolu s~párem lehkých řetězců poté tvoří globulární myosinové hlavy.

Tenký filament se skládá převážně z~aktinu. Aktin v~tenkém filamentu je polymerizován do dvou vláken stočených do alpha-šroubovice. Na této šroubovici jsou místa k~vázání myosinu. Tato místa jsou při relaxaci zakrytá tropomyosinem. Na něm jsou v pravidelných intervalech zavěšené komplexy troponinu. Jeho úkolem je při kontrakci navázat ionty vápníku a odstranit tropomyosin a dovolit navázání myosinových hlav na actin.

Každá myofibrila je pruhované vlákno kdy jednotlivé sekce se nazývají sarkomery. V~prostřed sarkomery se vyskytuje A-pásmo. Zde se prolínají actin a myosin. Ve středu A-pásma se nachází M-linie. Ty jsou tvořeny tmavě zbarvenými proteiny vázajícími jednotlivé molekuly myosinu k~sobě. Na hranicích sarkomery pak leží I-pásmo obsahující aktin. Uprostřed každého I-pásma se nachází Z-disk který ohraničuje konce jednotlivých sarkomer. \cite[costanzo2018physiology]

\midinsert
\clabel [sarkomera] {Schéma sarkomery}
\picw=7.5cm \cinspic assets/sarkomera.png
\caption/f Schéma sarkomery \cite[luther2009vertebrate]
\endinsert

\secc Motorické neurony
Ve svalu by nikdy neprobíhala kontrakce nebýt motorických neuronů. Motorické neurony jsou nervové buňky, které slouží k přenášení impulzů z kortexu mozku a mozkového kmene ke svalu. Dělí se na dva typy, horní a dolní.
Horní motorické neurony jsou součástí \glref{CNS} a vedou signál z kortexu mozku, mozkového kmene a mozečku míchou k jednotlivým dolním motorickým neuronům.

Dolní motorické neurony jsou poté nervové buňky, které mají za úkol přenášet signál od horních nervových neuronů. Existují tři hlavní typy dolních motorických neuronů: somatické motorické neurony, branchiální motorické neurony a viscerální motorické neurony. Somatické motorické neurony se dále dělí na tři podtypy: alfa, beta a gamma. Alfa motorické neurony inervují extrafuzální svalová vlákna a jsou primárními nosiči vzruchu při kontrakci kosterních svalů. Jejich těla leží v mozkovém kmeni či v míše. Gamma motorické neurony naopak inervují svalová vřeténka a určují jejich citlivost. \cite[zayia2023neuroanatomy]

\secc Kontrakce a relaxace svalového vlákna
V klidu jsou na myosinových hlavách připevněné molekuly \glref{ADP} a \glref{$\hbox{P}_i$}. Ty jsou záporně nabité. Stejně tak jsou záporně nabitá vlákna aktinu a tím pádem se myosin s aktinem slabě odpuzují. \cite[criswell2011cram]

Kontrakce začíná přijetím nervového akčního potenciálu, který se z dolních motorických neuronů šíří do T-tubul. Depolarizace T-tubul způsobí otevření $\hbox{Ca}^{++}$ kanálků v sarkoplazmatickém retikulu. Ty vypouštějí ionty $\hbox{Ca}^{++}$ které se navazují na troponin na tenkých filamentech, což posune tropomyosinem a odhalují se místa k vázání myosinu. Nyní začíná tzv. cross-bridge cyklus. Vypuštění \glref{ADP} a \glref{$\hbox{P}_i$} a myosinová hlava se přichytává k aktinu a následně se posouvá směrem k M-linii. To způsobuje pohyb aktinu a posun o cca 10 nm. Následně se na hlavu přichytává molekula \glref{ATP}, která se štěpí na \glref{ADP} a \glref{$\hbox{P}_i$}. Energie z reakce narovnává myosinovou hlavu do původní polohy a cyklus může začít znovu. \cite [costanzo2018physiology,criswell2011cram, whitneymenefee2020human]

Při relaxaci pak dochází ke snížení koncentrace $\hbox{Ca}^{++}$ za pomoci ATP-poháněných pump, kterého ionty odčerpávají zpět do sarkoplazmatického retikula, což způsobí opětovné navázání tropomyosinu na aktin a myosinová hlava se nemůže přichytit. Následně stejná polarita aktinu s \glref{ADP} způsobuje odsunutí filamentů do počáteční polohy. \cite [costanzo2018physiology,criswell2011cram, whitneymenefee2020human]

\midinsert
\clabel [kontrakce] {Kontrakce svalového vlákna}
\picw=7.5cm \cinspic assets/Skeletal_Muscle_Contraction.jpg
\rfc{Nechat?}
\caption/f Kontrakce svalového vlákna \cite[whitneymenefee2020human]
\endinsert

\sec Hardwarové zpracování signálu EMG
\rfc{vymyslet možná lepší název}
Přístroje pro měření \glref{EMG} měří změny potenciálu v jednotlivých svalech. Mohou být buď hodně obecné, či měřit konkrétní svalová vlákna. To záleží primárně na použité elektrodě a na filtrovaných frekvencích. Samotný přístroj je v podstatě voltmetr, který měří napětí velká řádově mikrovolty a následně je zesiluje na vhodné velikosti a potlačuje šum.

\secc Elektrody

Elektrody jsou způsob
